\documentclass[runningheads,a4paper]{llncs}

\usepackage{amssymb}
\setcounter{tocdepth}{3}
\usepackage{graphicx}
\usepackage[utf8]{inputenc}
\usepackage[T1]{fontenc}

\usepackage{url}



\begin{document}

\author{Jonas Lenze}

\section{Spielideen}

\subsection{Umsetzung virtueller Spiele in der physischen Welt}

Eine Reihe von Spielideen, die in der virtuellen Welt sehr populär sind.
\subsubsection{Snake}
Ein absoluter Klassiker, der früher auf keinem Handy fehlen durfte. Gerade wegen der
Popularität und Einfachheit recht interessant es in die physische Welt zu integrieren.
In einem begrenzten Areal gilt es eine Schlange geschickt zu steuern. Es müssen kleine
Items eingesammelt werden, wodurch die Schlange wächst. Außerdem muss darauf
geachtet werden, dass weder der Rand des Areals noch die eigene Schlange berührt wird.
\subsubsection{Capture the Flag}
Nicht nur in der virtuellen Welt Spielbar, aber wohl dort erst richtig bekannt geworden.
Auch hier ist ein recht einfacher Spielmechanismus ausschlaggebend für einen leichten
Einsatz in der physischen Welt.
Zwei Teams treten gegeneinander an. Jedes Team besitzt eine Flagge, dessen Standort für
das gegnerische Team bekannt ist. Es wird versucht die Flagge des jeweils anderen Teams
zum Standort der eigenen Flagge zu bringen.
\subsubsection{Domination}
Meist in Kriegsszenarien integrierter virtueller Spielmodi, der sich mit wenig Aufwand in die
physische Welt übertragen lässt.
Zwei Teams treten gegeneinander an. Jedes Team hat einen Punktestand, der zu Anfang
gleich ist. Sinkt der Punktestand auf Null, so ist das Spiel für dieses Team verloren. Es gibt
verschiedene Areale, die eingenommen werden können. Hat ein Team mehr Areale
eingenommen als das andere, läuft der Punktestand des Teams, das weniger Areale in
besitzt hat gegen Null.

\subsection{Integration einer virtuellen Welt in Bewegungsspiele}
Natürlich können wir die Integration auch von der anderen Seite betrachten. Es kann in
etablierte Bewegungsspiele der physischen Welt, eine virtuelle Umgebung eingefügt
werden.
Ein paar simple BeiSpiele:
\subsubsection{Verstecken}
Es gibt eine Person, die der Suchende ist. Alle weiteren Mitspieler verstecken sich möglichst
gut und versuchen vom Suchenden nicht entdeckt zu werden. Hat der suchende Spieler
alle Mitspieler gefunden ist das Spiel vorbei.
\subsubsection{Fangen}
Ein Spieler versucht alle anderen Mitspieler durch berühren zu „fangen“. Hat er geschafft
einen Mitspieler zu berühren ist nun dieser Mitspieler seinerseits dran einen anderen zu
fangen.
\subsubsection{Topfschlagen}
Ein Spieler bekommt die Augen verbunden. Er versucht anhand von Tipps der Mitspieler
(„näher dran“ oder „weiter weg“) einen Gegenstand zu finden.

\section{Spielkonzepte}
Einige Hilfsmittel, die zur Umsetzung der oben genannten BeiSpiele notwendig oder
hilfreich sind.

\subsection{Darstellung der physischen und virtuellen Umgebung [in einer Karte]}

Die physische Welt in eine virtuelle Umgebung zu übertragen kann oft sehr hilfreich sein.
Sei es in einer Karte als Übersicht oder lediglich eine Anzeige ob man sich innerhalb bzw.
außerhalb des Spielfelds befindet. Eine Karte hat insoweit den Vorteil, dass man dort auch
noch virtuelle Gegenstände einfügen kann, die in der physischen Welt nicht vorhanden
sind. Z.B. die nächsten Items bei Snake.

\subsection{Kompass}

Eine Richtungsangabe kann eine Karte ersetzen. Sei es um bei „Capture the Flag“ die
Richtung der Fahne des Gegners anzuzeigen oder für die „Snake“ das nächste
einzusammelnde Item. In der Regel hat es hier weniger Sinn, wenn die Kompassnadel nach
Norden zeigt.

\subsection{Akustische und haptische Orientierungshilfen}

Akustische oder haptische Signale können ebenso Hinweise geben auf in der Nähe
befindliche Interessengebiete (z.B. ertönen eines Signals oder Vibration bei erreichen eines
bestimmtem Umkreises von einem Item). Können aber auch als Bestätigung eingesetzt
werden, wenn z.B. etwas eingesammelt wurde.

\subsection{Chat}

Um mit seinen Teammitgliedern oder dem gegnerischen Team zu kommunizieren gibt es
eine Reihe von Möglichkeiten. Wenn man sich außer Sicht- und Hörweite befindet kann
dies über ein mobiles Gerät stattfinden. Beim Chat wird eine Nachricht verschickt, die der
Empfänger auf seinem Gerät einsehen kann und dann seinerseits eine Antwort verfassen
kann.

\subsection{Synchronisation zwischen mobilen Endgeräten}

Damit alle Mitspieler überhaupt miteinander Spielen können, ist es wichtig, dass die
einzelnen, sich im Spiel befindlichen mobilen Geräte, synchronisiert werden. Das heißt, das
jedes Gerät auf irgend eine Art und Weise mit den anderen Geräten kommuniziert. Es ist
wichtig, dass Daten, die jeder sehen kann, bei jedem identisch und zur selben Zeit
angezeigt werden.

\subsection{Kollision virtueller Objekte}

Treffen zwei virtuelle Objekte aufeinander muss in der Regel ein Event ausgelöst werden.
Wird bei Snake z.B. der eigene Schwanz berührt, was laut der Regeln nicht erlaubt ist, muss
dies dem Spieler mitgeteilt werden und evtl. weitere Ereignisse ausgeführt werden.

\subsection{Einsammeln von Objekten}
Eine Variante der Kollision mit virtuellen Objekten ist das Einsammeln. Wenn ein Spieler in
Reichweite eines Items ist, das es einzusammeln gilt, kann dies entweder automatisch
passieren oder über eine Aufforderung auf dem mobilen Gerät. Zur Bestätigung, dass
etwas eingesammelt wurde, kann nun wiederum ein akustisches oder haptisches Signal
gegeben werden.

\subsection{Geschwindigkeitsmessung}

Um das Spielgeschehen besser zu kontrollieren zu können, kann eine Messung der
Geschwindigkeit von Vorteil sein. Möchten wir z.B. bei Capute the Flag dem Fahnenträger
nicht erlauben eine gewissen Geschwindigkeit zu überschreiten, ist eine
Geschwindigkeitsmessung unabdingbar.


\subsection{Mensch-Maschine-Kommunikation}

Das komplette Spielgeschehen lebt nach der Integrierung von mobilen Endgeräten von der Kommunikation zwischen Mensch und Gerät. Wird auf dem Endgerät ein Kompass angezeigt, muss der Spieler insofern reagieren, dass er sich in die richtige Richtung dreht.
Wenn er etwas einsammeln möchte, kann es erforderlich sein, dass ein Button gedrückt
wird.


\end{document}

