\chapter{Fazit}
Es ist uns gelungen, zwei virtuelle Spiele umzusetzen, in dem die physische Umgebung der Teilnehmer als Spielfeld bzw. Map dient. Die Qualit�t des GPS-Sensors der verwendeten Handys kann hierbei jedoch die Steuerung ungenau erscheinen lassen und so den Spielspa� beeintr�chtigen. Zus�tzlich ist die Qualit�t der GPS-Daten von dem Wetter abh�ngig.
%So muss das Spielfeld gen�gend gro� sein und  Spieler m�ssen Acht geben keine zu scharfen Kurven zu laufen und d�rfen sich nicht zu schnell bewegen. 

In den momentan verwendeten Smartphones sind gravierend unterschiedlich pr�zise Sensoren verbaut. In Zukunft kann man auf eine gro�e Verbreitung zuverl�ssigerer Sensoren hoffen. Durch pr�zisere Positionsbestimmung k�nnten Spiele mit schneller Bewegung auf kleinerem Raum fl�ssig spielbar werden.

Die implementierten und angedachten Spiele vereinen eine virtuelle Welt mit der tats�chlichen Umgebung der Spieler. Infolgedessen m�ssen Spieler sowohl tats�chlichen wie virtuellen Objekten ausweichen. Daher ist es erforderlich, dass die Spieler st�ndig abwechselnd auf den Bildschirm ihres Smartphones und ihre Umgebung schauen. Die Spielgeschwindigkeit muss auch deshalb eher niedrig sein.