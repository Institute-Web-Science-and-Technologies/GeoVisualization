\chapter{Fazit}
Es ist uns gelungen, zwei virtuelle Mehrspieler-Spiele umzusetzen, in denen die physische Umgebung der Teilnehmer als Spielfeld dient und in Form einer Umgebungskarte auf den Smartphones der Mitspieler dargestellt wird. Dabei wird die dargestellte physische Umgebung durch virtuelle Objekte angereichert und die einzelnen Spieler interagieren direkt miteinander. 
%So muss das Spielfeld gen�gend gro� sein und  Spieler m�ssen Acht geben keine zu scharfen Kurven zu laufen und d�rfen sich nicht zu schnell bewegen. 

Die Qualit�t des GPS-Sensors der verwendeten Handys kann hierbei jedoch die Steuerung ungenau erscheinen lassen und so den Spielspa� beeintr�chtigen. Zus�tzlich ist die Qualit�t der GPS-Daten vom Wetter abh�ngig.
In den momentan verwendeten Smartphones sind zus�tzlich gravierend unterschiedlich pr�zise Sensoren verbaut. In Zukunft kann man auf eine gro�e Verbreitung zuverl�ssigerer Sensoren hoffen. Durch pr�zisere Positionsbestimmung k�nnten Spiele mit schneller Bewegung auf kleinerem Raum fl�ssig spielbar werden.

Die implementierten und die angedachten Spiele vereinen eine virtuelle Welt mit der tats�chlichen Umgebung der Spieler. Infolgedessen m�ssen Spieler sowohl tats�chlichen wie virtuellen Objekten ausweichen. Daher ist es erforderlich, dass die Spieler st�ndig abwechselnd auf den Bildschirm ihres Smartphones und ihre Umgebung schauen. Die Spielgeschwindigkeit sollte deshalb eher niedrig sein. Beim Testen der entwickelten Anwendungen hat sich herausgestellt, dass sie gut funktionieren, solange sich alle Spieler mit Schrittgeschwindigkeit bewegen. Laufen Spieler jedoch, funktionieren die Spielmechaniken nicht mehr. 

Die im Zuge dieses Projekts umgesetzten Spiele unterscheiden sich von den vorhandenen Smartphone-Spielen, welche die virtuelle Realit�t und tats�chliche Umgebung vereinen, vor allem durch h�heres Spieltempo und direkte Interaktion zwischen Spielern. Deswegen musste bei der Kollisionsabfrage R�cksicht auf die GPS-Genauigkeit genommen werden und es war n�tig als Regel festzulegen, dass Spieler nicht laufen d�rfen.
