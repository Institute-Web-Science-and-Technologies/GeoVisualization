%\section{Weitere Features}
F�r die Zukunft ist zu erwarten, dass mit Hilfe neuartiger Ger�te virtuelle und reine Realit�t noch mehr verschmelzen, als es heute der Fall ist. 
Einen vielversprechenden Ansatz bietet Google Glass, ein Minicomputer, der an eine Brille montiert Informationen in das Sichtfeld des Benutzers einblendet. Glass ist ein Paradebeispiel f�r augmented reality.
Die zweite und dritte Generation dieses Ger�tes sind derzeit (Stand Mai 2015) in der Entwicklung \cite{cb}.
Google Glass eignet sich hervorragend f�r die Umsetzung von Snake oder anderen Spielen in die reine Realit�t. Dadurch, dass das Ger�t am Kopf getragen wird und die Informationen direkt in das Sichtfeld eingeblendet werden, ist es nicht mehr n�tig w�hrend dem Spiel st�ndig auf das Smartphone zu schauen. Da das Spiel im Freien gespielt wird, erh�ht sich die Sicherheit des Nutzers wesentlich.
Des Weiteren k�nnen die Positionen von Wegpunkten und anderen virtuellen Objekten pr�ziser und dreidimensional in der Realit�t dargestellt werden, als es auf einem kleinen Smartphone Display der Fall ist. Zus�tzlich hat der Spieler die H�nde frei, was mehr M�glichkeiten bietet, wie er mit den virtuellen Objekten interagieren kann.

Microsoft HoloLens \footnote{\url{https://www.microsoft.com/microsoft-hololens/en-us/experience}} ist ein weiteres Ger�t, das virtuelle und reine Realit�t vermischt. Der Ansatz ist aber derzeit nur auf geschlossene R�ume beschr�nkt.
Hologramme werden in das Sichtfeld des Benutzer eingeblendet, so dass man mit virtuellen und realen Objekten ganz nat�rlich und gleichzeitig interagieren kann. Microsoft wirbt damit, sowohl den Alltag zu Hause, als auch das produktive Arbeiten zu vereinfachen.
F�r das Snake-Spiel im Freien ist HoloLens derzeit nicht geeignet. Sollte es aber in Serienproduktion gehen, ergeben sich eine ganze F�lle an M�glichkeiten f�r innovative Spiele.

Oculus Rift \footnote{\url{https://www.oculus.com/}} ist ein Ger�t f�r die vollst�ndig visuelle virtuelle Realit�t. An die Augen gelangen keine realen Eindr�cke mehr. Rift ist ungeeignet f�r ein Spiel im Freien. Google stellt aber im Rahmen seiner Maps und Street View Dienste dreidimensionale Ansichten von St�dten und Sehensw�rdigkeiten bereit \footnote{\url{https://support.google.com/maps/answer/3092441?hl=de}}.
Hier k�nnte es in Zukunft m�glich sein, solche Daten in Anwendungen f�r Rift zu verarbeiten. So w�re es m�glich, sich im Snake-Spiel virtuell durch reale Orte zu bewegen, w�hrend man zu Hause im Sessel sitzt.