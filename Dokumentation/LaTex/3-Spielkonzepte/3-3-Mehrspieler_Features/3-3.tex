\section{Mehrspieler-Features}
Die folgenden Konzepte erm�glichen das Spielen mit mehreren Spielern gleichzeitig  und Interaktionen zwischen diesen.

\subsection*{Synchronisation zwischen mobilen Endger�ten}\label{sec:synchronisation-zwischen-mobilen-endgeraeten}

Damit alle Spieler miteinander spielen k�nnen, ist es wichtig, dass sie miteinander, entweder direkt oder �ber einen Server, kommunizieren k�nnen. Zudem ist es f�r die meisten Spiele erforderlich, dass die einzelnen sich im Spiel befindlichen mobilen Ger�te synchronisiert werden. Das hei�t, dass mindestens eine Instanz die Position aller teilnehmenden Ger�te und ihren zu teilenden Informationen kennt. Es ist wichtig, dass Daten, die jeder sehen kann, bei jedem identisch und zur selben Zeit angezeigt werden.
Um dies verwirklichen zu k�nnen ist z.B. eine Server-Client-Kommunikation (siehe Abschnitt \ref{kommunikation}) erforderlich.


\subsection*{Chat}\label{sec:chat}
Um mit seinen Teammitgliedern oder dem gegnerischen Team zu kommunizieren gibt es eine Reihe von M�glichkeiten. Wenn man sich au�er Sicht- und H�rweite befindet, kann dies durch mobile Ger�te stattfinden. Zum Beispiel indem Nachrichten versendet werden, die f�r alle Chatgruppenteilnehmer (beispielsweise f�r das eigene Team, oder f�r alle Mitspieler) zu sehen sind.
Auch hierf�r ist eine Server-Client-Kommunikation (siehe Abschnitt \ref{kommunikation}) und zur Darstellung die GUI (siehe Abschnitt \ref{gui}) erforderlich.