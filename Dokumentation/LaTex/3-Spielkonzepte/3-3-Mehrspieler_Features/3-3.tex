\section{Mehrspieler-Features}
Konzepte, die es erm�glichen mit mehreren Spielern gleichzeitig zu spielen, bzw. Interaktionen zwischen ihnen schaffen.

\subsubsection{Synchronisation zwischen mobilen Endger�ten}\label{sec:synchronisation-zwischen-mobilen-endgeraeten}

Damit alle Spieler miteinander spielen k�nnen, ist es wichtig, dass sie miteinander entweder direkt oder �ber einen Server kommunizieren k�nnen. Zudem ist es f�r die meisten Spiele erforderlich, dass die einzelnen, sich im Spiel befindlichen mobilen Ger�te synchronisiert werden. Das hei�t, dass mindestens eine Instanz die Position aller teilnehmenden Ger�te und ihren zu teilenden Informationen kennt. Es ist wichtig, dass Daten, die jeder sehen kann, bei jedem identisch und zur selben Zeit angezeigt werden.

Um dies verwirklichen zu k�nnen ist eine Server-Client-Kommunikation (s. \ref{kommunikation}) erforderlich.


\subsubsection{Chat}\label{sec:chat}
Um mit seinen Teammitgliedern oder dem gegnerischen Team zu kommunizieren gibt es eine Reihe von M�glichkeiten. Wenn man sich au�er Sicht- und H�rweite befindet, kann dies durch mobile Ger�te stattfinden, z.B. indem Nachrichten versendet werden, die f�r alle Chatgruppenteilnehmer (beispielsweise f�r das eigene Team, oder f�r alle Mitspieler) zu sehen sind.

Auch hierf�r wird die Server-Client-Kommunikation (s. \ref{kommunikation}) gebraucht und zur Darstellung die  GUI (s. \ref{gui}).