\chapter{Einleitung}
%Wir haben uns damit besch�ftigt Interaktivit�t zwischen Smartphones und den Benutzern herzustellen. Dazu nutzen wir verschiedene Sensorik, die uns solche Ger�te bieten.

%Was bedeutet Interaktivit�t f�r uns in diesem Zusammenhang? Wir wollen es schaffen die physische Umgebung in eine virtuelle Welt einflie�en zu lassen. Umgekehrt sollen Entscheidungen in der Realit�t von virtuellen Ereignissen unterst�tzt werden.


Durch die fortlaufende technologische Entwicklung vergr��ern sich M�glichkeiten und Qualit�t virtueller Realit�ten (\textit{virtual reality}, VR) stetig.
W�hrend VR meist f�r das Eintauchen in komplett synthetisch erzeugte Welten steht, ist \textit{augmented reality} (in etwa \glqq erweiterte Realit�t\grqq) eine Spielart der VR, bei der virtuelle Elemente in die tats�chliche Welt integriert werden \cite{azuma}. 
Smartphones bieten durch Ortungsfunktionalit�t und andere Sensorik die M�glichkeit, ihre Umgebung zu erfassen und durch virtuelle Elemente zu erweitern. Dadurch lassen sich interaktive Bewegungsspiele entwickeln, bei denen jeder Spieler durch sein Smartphone gleichzeitig geortet und mit Informationen �ber den Spielverlauf versorgt werden kann. Mobiles Internet macht es zus�tzlich m�glich, dass sich Smartphones der Mitspieler und Mitspielerinnen synchronisieren. Eine solche Interaktivit�t zwischen maschinengest�tztem Spiel und physischer Umgebung ist weder mit einem station�ren Rechner, noch mit einem nicht internetf�higen Handy m�glich.
Mehrspieler-Anwendungen, die versuchen eine erweiterte Realit�t mit dem Smartphone zu erzeugen sind vorhanden (siehe Abschnitt \ref{related}). Jedoch setzen nur wenige direkte Interaktion zwischen den Spielern um und erfordern keine st�ndige Aufmerksamkeit und Bewegung von den Spielern.


In diesem Praktikum wird Android als Plattform genutzt, um die Beispiel-App\-li\-ka\-tio\-nen (siehe Kapitel \ref{implementationen}) umzusetzen.  Alternativen w�ren iOS oder Windows Phone. Zweiteres hat einen zu geringen Marktanteil \cite{android} und die Entwicklung f�r iOS ist mit hohen Lizenzkosten verbunden.  

In den folgenden Kapiteln werden zun�chst verschiedene Ideen vorgestellt, wie man virtuelle Spiele in die physische Welt integrieren kann oder eine virtuelle Welt in traditionelle Bewegungsspiele eingebaut werden kann (siehe Kapitel \ref{ideen}). Danach werden Spielkonzepte genannt, die zur Umsetzung dieser Ideen ben�tigt werden (siehe Kapitel \ref{konzepte}). Es wird untersucht, wie sich diese Konzepte technisch umsetzen lassen (siehe Kapitel \ref{technisch}). Mit Hilfe der erarbeiteten technischen L�sungen werden einige Spielideen als Android-Apps umgesetzt (siehe Kapitel \ref{implementationen}).

