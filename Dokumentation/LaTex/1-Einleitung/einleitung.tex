\chapter{Einleitung}
{\color{red}ein Satz, der das Thema der Arbeit paraphrasiert}

Was bedeutet Interaktivit�t f�r uns in diesem Zusammenhang? Wir wollen es schaffen die physische Umgebung in eine virtuelle Welt einflie�en zu lassen. Umgekehrt sollen Entscheidungen in der Realit�t von virtuellen Ereignissen beeinflusst werden.
Smartphones bieten durch Sensorik und interaktiver Kommunikation die Grundlage hierzu. So ist es m�glich orts�bergreifend mit dem Spielpartner zu interagieren. Dabei ist man auch ortsunabh�ngig im Austausch von Daten durch das Mobilfunknetz. 
{\color{red}Def. virtuelle Realit�t}
Eine weitere Auspr�gung w�re die sogenannte �Augmented Reality. {\color{red}zuerst ein Satz, der AR definiert, Unterschied zur VR, dann konkrete Umsetzung auf dem Handy (folgenden Satz umformulieren)} Hierbei werden meist in dem dargestellten Bild einer Smartphone-Kamera zus�tzliche optische Informationen eingeblendet.

{\color{red}Welche zus�tzlichen (auch nicht umgesetzte) Features bieten Smartphones vs Computerspiele vs "Kinderspiele"/Bewegungsspiele ? wie kann man das Smartphone in einem solchen Spiel einsetzen? was ist der Mehrwert Smartphone einzusetzen?

Was gibts schon (Geocashing, Ingress) ? haben wir was neu gemacht ?}

Warum haben wir uns f�r Android als Plattform entschieden? Die anderen Alternativen w�ren iOS und Windows Phone gewesen. Letzteres hat einen zu geringen Marktanteil {\color{red}Quelle} und die Entwicklung f�r iOS kommt mit hohen Lizenzkosten daher. 

{\color{red}�berblick folgende Kapitel (einfach nennen)}

\newpage
