\chapter{Einleitung}
Wir haben uns damit beschäftigt Interaktivität zwischen Smartphones und dem Benutzer herzustellen. Dazu nutzen wir verschiedene Sensorik, die uns solche Geräte bieten.

Was bedeutet Interaktivität für uns in diesem Zusammenhang? Wir wollen es schaffen die physische Umgebung in eine virtuelle Welt einfließen zu lassen. Umgekehrt sollen Entscheidungen in der Realität von virtuellen Ereignissen beeinflusst werden.
Smartphones bieten durch Sensorik und interaktiver Kommunikation die Grundlage hierzu. So ist es möglich ortsübergreifend mit dem Spielpartner oder der Spielpartnerin zu interagieren. Dabei ist man auch ortsunabhängig im Austausch von Daten durch das Mobilfunknetz. 
Virtuelle Realität bedeutet eine durch den Computer geschaffene Welt, in der Menschen mit dieser interagieren.
Eine weitere Ausprägung wäre die sogenannte “Augmented Reality”. Der Unterschied zur Virtuellen Realität besteht darin, dass der Nutzer sich in der wirklichen Welt bewegt und diese mit virtuellen Informationen befüllt wird. So werden z.B. in dem dargestellten Bild einer Smartphone-Kamera zusätzliche optische Informationen eingeblendet.

Bekannte Anwendungen in diesem Gebiet sind Geocaching oder Google Ingress. Doch welche Unterschiede bietet unser Projekt dazu?
Zum einen haben wir eine Kollisionsabfrage implementiert, die Nutzer dazu motiviert die eigene Position und die der anderen zu verfolgen. Auch muss man ständig in Bewegung bleiben, damit man nicht mit sich selbst kollidiert. Also haben wir eine Echtzeit Interaktivität erschaffen, wobei der Nutzer ständig aufmerksam sein muss. 

Betrachten wir Smartphone Apps im Unterschied zu klassischen Computerspielen. Beim letzteren ist man statisch und hat keine Interaktion mit der Umgebung. Eine Smartphone App bietet genau das. Der Nutzer ist nicht an den Ort gebunden und hat seine Apps immer mit dabei. 
Zwar bietet ein Computerspiel auch Interaktivität mit anderen Nutzern, wenn man z.B. Online-Spiele nimmt, allerdings bietet der Computer keine Sensorik um sich in der Augmented Reality zu bewegen.

Warum haben wir uns für Android als Plattform entschieden? Die anderen Alternativen wären iOS und Windows Phone gewesen. Letzteres hat einen zu geringen Marktanteil \cite{android} und die Entwicklung für iOS kommt mit hohen Lizenzkosten daher.  

Im folgenden beschäftigen wir uns mit den Spielideen, den Konzepten, den technischen Lösungen und der Implementation von Beispiel-Applikationen. Zudem ziehen wir ein Fazit und geben einen kleinen Ausblick.

\newpage
