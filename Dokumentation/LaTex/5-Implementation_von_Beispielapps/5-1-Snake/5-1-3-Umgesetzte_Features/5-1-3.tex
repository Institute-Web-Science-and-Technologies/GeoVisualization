%\subsection{Umgesetzte Spielkonzepte}
%
%Es ist uns gelungen viele der angedachten Spielkonzepte im Zuge der Implementation von Snake umzusetzen.
%Durch die Karte, die die Umgebung darstellt und dabei virtuelle wie reelle Objekte einbezieht, kommen mehrere Features zum Einsatz. Umgesetzt worden ist die Integration virtueller Objekte in die physische Umgebung (siehe \ref {sec:integration-virtueller-objekte-in-die-physische-umgebung}) und die Darstellung der physischen und virtuellen Umgebung (siehe \ref{sec:darstellung-der-physischen-und-virtuellen-umgebung}).
%F�r die Spiellogik von Snake musste die Kollision virtueller Objekte  (siehe \ref{sec:kollision-virtueller-objekte}) und das Einsammeln von Objekten (siehe \ref{sec:einsammeln-von-objekten}) implementiert werden, wobei die Objekte konkret als Schlangen und H�hner ausgepr�gt sind.
%Die Mensch-Maschine-Kommunikation (siehe \ref{sec:mensch-maschine-kommunikation}) von Smartphone zum Spieler funktioniert bei dieser Anwendung �ber die graphische Ausgabe des Spielzustands mit Hilfe der Umgebungskarte. Spieler hingegen beeinflussen den Spielzustand ausschlie�lich, indem sie sich und damit ihr Handy bewegen.
%�ber den implementierten Chat (siehe \ref{sec:chat}) k�nnen sich Spieler austauschen.
%Die einzelnen Clients synchronisieren ihren Zustand �ber einen speziellen Server (siehe \ref{sec:synchronisation-zwischen-mobilen-endgeraeten}).
%
%
