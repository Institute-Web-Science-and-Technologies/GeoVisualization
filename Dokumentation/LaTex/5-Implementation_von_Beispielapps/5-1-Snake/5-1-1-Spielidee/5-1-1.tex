\subsection{Spielidee}
Als erstes Spiel, das wir umsetzen, haben wir uns für Snake entschieden. Die Grundregeln von Snake sind nahezu jedem bekannt, da es auf allen Nokia Handys der 90er Jahre mitgeliefert wurde. Des weiteren lässt sich das Spiel
relativ leicht von der virtuellen Welt in die physische übertragen, da die Steuerung nur auf der Bewegung des Kopfes der Schlange basiert, was man in der Realität durch die Bewegung des Spielers steuern kann. Wir haben uns
für eine Mehrspieler Variante entschieden bei der in einem Spiel mehrere Zielpunkte gleichzeitig angezeigt werden, die bei Berührung den Punktestand des Spielers erhöhen und die Schlange etwas vergrößern. Ziel des Spiels ist es schneller als die anderen eine einstellbare Anzahl von Punkten einzusammeln. Wenn man mit sich selbst oder einer anderen Schlange kollidiert wird der eigene Punktestand und die Länge der Schlange zurückgesetzt.