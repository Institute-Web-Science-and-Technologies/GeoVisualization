\subsection{Spielidee}
Das erste Spiel, das wir umsetzen, ist das schon mehrfach erw�hnte Snake. Die Grundregeln von Snake sind nahezu jedem bekannt, da es auf nahezu allen Nokia-Handys der 90er Jahre mitgeliefert wurde.
In der implementierten Beispielapplikation verk�rpert jeder Spieler  eine Schlange. Der Kopf der virtuellen Schlange befindet sich auf der Spielerposition. Sobald der Spieler sich anf�ngt zu bewegen, zieht er einen Schwanz hinter sich her, der aus mehreren Gliedern besteht. Die Anzahl der Glieder und damit die L�nge des Schwanzes entspricht dem Punktestand des Spielers. Der Spieler muss versuchen, mit dem Kopf zuf�llig platzierte Bonusobjekte (H�hner/ Chickens) einzusammeln und muss gleichzeitig vermeiden mit seinem eigenen Schwanz oder einer anderen Schlange zu kollidieren.
 Des weiteren l�sst sich das Spiel relativ leicht aus der virtuellen Landschaft in die physische Welt �bertragen, da die Steuerung nur auf der Bewegung des Kopfes der Schlange basiert. Dies kann in der Realit�t durch die Bewegung des Spielers selbst gesteuert werden. Wir haben uns f�r eine Mehrspieler Variante entschieden bei der in einem Spiel mehrere Zielpunkte (Chickens) gleichzeitig angezeigt werden, die beim (virtuellen) Einsammeln den Punktestand des Spielers erh�hen und die Schlange vergr��ern. Ziel des Spiels ist es schneller als die gegnerischen Schlangen eine einstellbare Anzahl von Punkten zu erreichen. Wenn man mit sich selbst oder einer anderen Schlange kollidiert wird der eigene Punktestand und die L�nge der Schlange zur�ckgesetzt.

