\subsection{Spielidee}
Als erstes Spiel, das wir umsetzen, haben wir uns f�r Snake entschieden. Die Grundregeln von Snake sind nahezu jedem bekannt, da es auf allen Nokia Handys der 90er Jahre mitgeliefert wurde.
Jeder Spieler verk�rpert eine Schlange. Der Kopf der virtuellen Schlange befindet sich auf der Spielerposition, dahinter zieht sie einen Schwanz her, der sich aus mehreren Gliedern zusammensetzt. Die Anzahl der Glieder und damit die L�nge des Schwanzes entspricht dem Punktestand des Spielers. Der Spieler muss versuchen, mit dem Kopf zuf�llig platzierte Bonusobjekte (H�hner) einzusammeln und muss gleichzeitig vermeiden mit dem eigenen Schwanz oder einer anderen Schlange zu kollidieren.
 Des weiteren l�sst sich das Spiel
relativ leicht von der virtuellen Welt in die physische �bertragen, da die Steuerung nur auf der Bewegung des Kopfes der Schlange basiert, was man in der Realit�t durch die Bewegung des Spielers steuern kann. Wir haben uns
f�r eine Mehrspieler Variante entschieden bei der in einem Spiel mehrere Zielpunkte gleichzeitig angezeigt werden, die bei Ber�hrung den Punktestand des Spielers erh�hen und die Schlange vergr��ern. Ziel des Spiels ist es schneller als die anderen eine einstellbare Anzahl von Punkten einzusammeln. Wenn man mit sich selbst oder einer anderen Schlange kollidiert wird der eigene Punktestand und die L�nge der Schlange zur�ckgesetzt.

