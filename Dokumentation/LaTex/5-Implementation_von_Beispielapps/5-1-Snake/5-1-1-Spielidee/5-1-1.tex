Zwei der vorgestellten Spielideen werden mit Hilfe der ben�tigten Features als Android-App implementiert. Hierzu wird ein Framework entwickelt, das allgemein ben�tigte Funktionalit�ten bereitstellt, und darauf aufbauend werden die beiden Beispielspiele umgesetzt.



\section{Allgemeines Framework}
%\subsection{Spiellogik}
Unser System besteht aus zwei Komponenten. Eine Android-App mit der gesamten Spiellogik und einem Server �ber den die Kommunikation zwischen den mobilen Endger�ten abl�uft.


\subsection*{Server}
Der implementierte Server arbeitet mit Java und ZeroMQ. Er �ffnet zwei Kommunikationskan�le mit dem Client. Der erste ist ein ZeroMQ-Request-Reply-Socket-Paar, �ber das die Endger�te Nachrichten an den Server senden. Der zweite ist ein Publish-Subscribe-Socket-Paar, �ber das die Nachrichten an Gruppen von Endger�ten weitergeleitet werden. Eine Nachricht besteht aus drei Teilen: einer Adresse, dem Nachrichtentyp und einem serialisierten Objekt vom Typ TransferObject (siehe Abbildung \ref{fig:transfer}). Die Adresse ist entweder die ID eines Spielers oder die ID einer Spielinstanz. Wir nutzen ZeroMQs Multipart-Message-Feature um diese Teile voneinander getrennt bei der Kommunikation zu �bermitteln. Der Server sendet eingehende Nachrichten an bestimmte Clients weiter. Das kann entweder ein einzelner Client, oder alle Clients die sich in einer Spielsession befinden, sein. Der Server kennt dabei den Zustand der einzelnen Spiele nicht. Auf dem Server befindet sich eine Liste von momentan aktiven Spielen. Wenn eine Nachricht vom Typ "`create\_game"' empfangen wird tr�gt der Server die ID des Spiels in eine Liste ein. Diese Liste wird einem Client gesendet wenn er sie �ber eine entsprechende Nachricht anfragt. Dabei werden vorher Spiele aus der Liste gel�scht, bei denen in den letzten 5 Minuten keine Nachricht auf dem Server empfangen wurde.





%
%\begin{figure}
%	\begin{center}
%		\includegraphics[width=0.5\textwidth]{img/transferobject.png}
%		
%		\caption{UML-Diagramm der Klasse\newline geoviz.communication.TransferObject}
%		\label{fig:transfer}
%	\end{center}
%\end{figure}


\subsection*{Client}

%Unsere Android-Applikation besteht aus drei Fragmenten zwischen denen man durch Wischen wechseln kann. Das erste Fragment ist ein Chat �ber den Spieler Textnachrichten an ihre Mitspieler im gleichen Spiel senden k�nnen.


Im Hintergrund laufen zwei Threads f�r die Kommunikation. Der eine k�mmert sich um den Empfang von Nachrichten und beinhaltet einen ZeroMQ-Subscriber-Socket. Dieser empf�ngt alle Nachrichten, die vom Publisher"=Socket auf dem Server direkt an die ID des Clients oder die ID des Spiels, in dem er sich befindet, adressiert sind. Der zweite Thread sendet Nachrichten an den Server. 
Bei einem Spielbeitritt einer laufenden Session, wird �ber den Server eine Anfrage des aktuellen Status des Spiels an alle momentanen Mitspieler gesendet, welche den Zustand des Spiels alle an den anfragenden Client zur�cksenden. Dieser beachtet nur die erste Nachricht und verwirft den Rest.
Des weiteren l�uft ein LocationClient, welcher immer die aktuelle Position als Nachricht an alle Spieler der selben Spielinstanz weiterschickt. 

\subsection*{GUI}\label{gui}

%Die GUI (\textit{graphical user interface}) wurde mit Android und den GooglePlay Services (einschlie�lich Google Maps) umgesetzt. Die f�r den Benutzer sichtbaren Bildschirme werden in der Entwicklung mit Android in Activities\footnote{\url{https://developer.android.com/guide/components/activities.html}} organisiert. Diese enthalten Informationen �ber das Layout, sowie Funktionalit�ten des Bildschirms.Die mitgelieferten M�g\-lich\-kei\-ten des Android SDK sind f�r die GUI-Umsetzung dieses Projektes vollkommen zufriedenstellend. 

Die GUI setzt sich aus einem Login-Screen und einem Swipe-Screen zusammen, welcher entsprechend dem erstelltem oder beigetretenem Spiel geladen wird.

\paragraph{Login-Screen:}
Diese Activity wird zuerst aufgerufen und zeigt einen Bildschirm auf dem der Spieler einen Benutzernamen eingibt und danach mit dem
"`Start"' Button zur n�chsten Activity wechselt, welche das eigentliche Spiel zeigt (siehe Abbildung \ref{fig:chat}).
%Bild wird unter swipe-screen angezeigt

\begin{figure}[t]
    \includegraphics[width=0.5\textwidth]{4-Technische_Loesungen/4-5-GUI/Data/login_screen.png}
    \includegraphics[width=0.5\textwidth]{4-Technische_Loesungen/4-5-GUI/Data/chat_screen.png}
     \caption{Login-Screen (links) und Chat-Screen (rechts)}
     \label{fig:chat}
\end{figure}


\paragraph{Swipe-Screen:}
Schon in der fr�hen Entwicklungsphase war festzustellen, dass die verschiedenen Elemente der GUI  zu zahlreich sind, um sie auf einen Bildschirm umzusetzen. Die eigentliche Kartendarstellung w�re sonst zu klein gewesen. Also wurde entschieden die verschiedenen Elemente auf weitere Bildschirme zu verteilen. In den ersten Entw�rfen geschah dies �ber einzelne Activities, also mehrere Voll\-bild-Fens\-ter. Um gewisse Android Kom\-fort-Funk\-tio\-nen und Gesten dem Nutzer zur Verf�gung zu stellen, wurden die zun�chst eigenst�ndigen Activites zu Fragments \footnote{\url{https://developer.android.com/guide/components/fragments.html}} umgebaut. Fragments kann man als "`Sub-Activity"' verstehen. Diese verf�gen jeweils �ber ein eigenes Layout und k�nnen eigene Funktionalit�ten enthalten, m�ssen aber bei Verwendung immer einer Activity zugewiesen werden. Anschlie�end wurden sie in einem so genannten "`Swipe-Screen"' zusammengefasst. In diesem wurden die Fragments als Tabs organisiert. Der User kann nun entweder durch "`wischen"' (\textit{swipe}) oder durch klicken auf die Tabs durch die GUI Navigieren. 

Ein weiterer Vorteil ist ebenfalls, dass benachbarte Tabs jeweils vorgeladen (Laden der Widgets) und noch im Speicher erhalten bleiben. Damit wird sichergestellt, dass der Tab-Wechsel durch Wischen "`geschmeidig"' abl�uft. Dadurch ergeben sich auch geringere Ladezeiten zwischen den einzelnen Bildschirmen. Zudem ist die  "`Wiederverwendbarkeit"' von Fragments als UI (\textit{user interface}) ebenfalls n�tzlich, wenn weitere �hnliche Spiele umgesetzt werden sollen. Der Swipe-Screen umfasst drei Tabs: Der Map-Screen zeigt den aktuellen Spielzustand, �ber den Chat-Screen k�nnen sich Spieler einer Spielinstanz austauschen und der Game-Screen erlaubt es, eine neue Spielinstanz zu erstellen oder einer bestehenden Spielinstanz beizutreten. 

F�r jeden Spiel-Modus ist ein separater SwipeScreen vorgesehen. Falls m�glich werden Fragmente, an denen nichts ge�ndert werden muss, geladen (z.B. Chat-Screen-Fragment oder Game-Screen-Fragment). Im Fall des Map-Screen-Fragment wird ein Komplett neues Fragment mit angepasstem Layout (z.B. weitere Buttons oder Anzeigen) geladen.  

\paragraph{Chat-Screen:}
%erkl�r IRC, das kommt aus dem Nichts
In diesem Fragment wird der Chat  dargestellt, um kurze Textnachrichten anderen Spielern zu senden oder von ihnen zu empfangen (siehe Abbildung \ref{fig:chat}). Die grafische Umsetzung des Chats ist simpel gew�hlt um m�glichst viele Nachrichten anzeigen zu k�nnen. Es wird der jeweilige Benutzername, Uhrzeit und die eigentliche Nachricht angezeigt. Die Eingabe der Chat-Nachricht erfolgt in einem Text-Eingabe-Feld. Die Anzeige der Chat-Nachrichten erfolgt in einem einfachen Textanzeige-Feld (TextView\footnote{{\url{http://developer.android.com/reference/android/widget/TextView.html}}}), das wiederum in einem scrollbarem Feld (SrcollView\footnote{\url{http://developer.android.com/reference/android/widget/ScrollView.html}}) liegt. Hierdurch ist es m�glich den Nachrichtenverlauf zu durchsuchen. Wird eine Chat-Nachricht  empfangen, wird diese an das Textfeld angeh�ngt. Hierbei ist zu beachten, dass die selbst verschickten Nachrichten erst an den Server gesendet werden und dann jeweils an die entsprechenden Nutzer. Dabei muss eine geringe �ber\-tra\-gungs\-ver\-z�\-ge\-rung in Kauf genommen werden, jedoch ist die korrekte Reihenfolge der Nachrichten ge\-w�hr\-leis\-tet.


\begin{figure}[t]
    %\includegraphics[width=0.5\textwidth]{4-Technische_Loesungen/4-5-GUI/Data/map_screen_snake.png}
    \includegraphics[width=0.5\textwidth]{5-Implementation_von_Beispielapps/5-1-Snake/Data/map_screen_snake.png}
    \includegraphics[width=0.5\textwidth]{4-Technische_Loesungen/4-5-GUI/Data/game_screen.png}
         \caption{Map-Screen des Spiels Snake (links) und Game-Screen (rechts)}
         \label{fig:game}
\end{figure}

\paragraph{Map-Screen:}
In diesem Fragment wird die Karte der n�heren Umgebung des Spielers dargestellt (siehe Abbildung \ref{fig:game}). Auf der Karte werden alle virtuellen Objekte und Mitspieler angezeigt, die in der aktuellen Spielinstanz vorhandenen sind und f�r den Spieler sichtbar sein sollen. Je nachdem welches Spiel gespielt wird sind eventuell zus�tzliche In\-ter\-ak\-tions\-m�g\-lich\-kei\-ten und Anzeigefelder vorhanden, wie beispielsweise die Anzeige des Punktestands der Teams.


\paragraph{Game-Screen:}
In diesem Fragment werden momentan aktive Spielinstanzen angezeigt (siehe Abbildung \ref{fig:game}). Auch ist es m�glich neue Spiele zu erstellen. Die Anzeige der Liste der Spiele erfolgt in einer scrollbaren Liste (ListView). Diese wird regelm��ig aktualisiert. Die Listen-Elemente sind interaktiv: Ein Klick auf das entsprechende Spiel startet den Beitritt. Um ein neues Spiel eines gewissen Typs zu erstellen w�hlt man in einen Drop\-down-Men� (bei Android Spinner) den entsprechenden Modus aus und klickt auf \glqq Create\grqq. Man selbst betritt dieses Spiel und der Eintrag in der Spiele-Liste wird vorgenommen. 

%\subsection*{Haptisches Feedback}
%Die graphische Benutzeroberfl�che kann durch Haptisches Feedback erg�nzt werden. Haptisch bedeutet \glqq f�hlbar\grqq oder \glqq ber�hrbar\glqq. Bei Smartphones betrachten wir in diesem Zusammenhang den Vibrationsalarm. Dieser wird nativ dazu genutzt um dem Nutzer mitzuteilen, dass ein Anruf oder eine Nachricht eingegangen ist oder um Eingaben zu best�tigen.
%Erzeugt werden die Vibrationen durch einen kleinen Motor, der durch eine Unwucht daf�r sorgt, dass das Geh�use vibriert. Bei einigen Ger�ten wird sogar der Lautsprecher dazu genutzt, indem er niedrigfrequente T�ne erzeugt, die das Geh�use in Schwingung versetzen.





%Das zweite Fragment ist die Karte auf der wir unser Spiel darstellen.  
%F�r die Darstellung der Karte verwenden wir GoogleMaps. 

%Im dritten Fragment kann ein Spieler eine neue Spielinstanz erstellen oder einem bereits laufendem Spiel beitreten. Die Liste der momentan laufenden Spiele wird per Knopfdruck vom Server abgefragt. Wenn man einer laufenden Session beitritt wird �ber den Server eine Anfrage des aktuellen Status des Spiels an alle momentanen Mitspieler gesendet, welche den Zustand des Spiels alle an den anfragenden Client zur�ck senden. Dieser beachtet nur die erste Nachricht und verwirft den Rest.



 


