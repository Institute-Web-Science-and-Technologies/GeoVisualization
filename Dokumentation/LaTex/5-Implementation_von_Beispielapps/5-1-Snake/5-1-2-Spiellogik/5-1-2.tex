\subsection{Spiellogik}
Unser System besteht aus 2 Komponenten. Eine Android-App mit der gesamten Spiellogik und einem Server �ber den die Kommunikation zwischen den mobilen Endger�ten abl�uft.

\subsection{Server}
Unser Server arbeitet mit Java und ZeroMQ. Er �ffnet zwei Kommunikationskan�le mit dem Client. Der erste ist ein ZeroMQ Request Reply Socket Paar �ber das die Endger�te Nachrichten an den Server senden. Der zweite ist ein Publish Subscribe Socket Paar �ber das die Nachrichten m�glichst schnell an Gruppen von Endger�ten weitergeleitet werden. Eine Nachricht besteht bei uns aus 3 Teilen: Der Empf�nger der Nachricht,  der Nachrichtentyp und der eigentlichen Nachricht. Wir nutzen ZeroMQs multipart Message Feature um diese Teile voneinander getrennt bei der Kommunikation zu �bermitteln. Haupts�chlich sendet der Server Nachrichten einfach an den Empf�nger, welcher entweder ein einzelner Client oder alle Clients die sich in eine Spielsession befinden ist, weiter. Des weiteren nutzen wir den Server um die Gamesessions zu verwalten. Wenn eine Nachricht vom Typ create\_game empfangen wird schreibt der Server die ID des Spiels in eine Liste ein. Diese Liste wird einem Client gesendet wenn er sie �ber eine Nachricht vom Typ Get\_Gamelist anfragt.

\subsection{Client}
Unsere Android Applikation besteht aus 3 Fragmenten zwischen denen man durch wischen wechseln kann. Das erste Fragment ist ein kleiner Chat �ber den Spieler Textnachrichten an ihre Mitspieler im gleichen Spiel senden k�nnen. {Hier sollte ein Screenshot hin das verbraucht Platz und so}
Das zweite Fragment ist die Karte auf der wir unser Spiel darstellen. {Hier sollte ein Screenshot hin} 
F�r die Darstellung der Karte verwenden wir GoogleMaps. 
Im dritten Fragment kann ein Spieler eine neue Spielsession erstellen oder einem bereits laufendem Spiel beitreten. Die Liste der momentan laufenden Spiele wird per Knopfdruck vom Server abgefragt. {Noch ein dritter Screenshot oder so}
Im Hintergrund laufen zwei Threads f�r die Kommunikation. Der eine k�mmert sich um den Empfang von Nachrichten und beinhaltet ein ZeroMQ Subscriber Socket welches vom Publisher auf dem Server alle Nachrichten, die direkt an die ID des Clients oder die ID des Spiels in dem er sich befindet adressiert sind, empf�ngt. Der zweite sendet Nachrichten an den Server. Des weiteren l�uft ein LocationClient welcher immer wenn ein Positionsupdate reinkommt die neue Position als Nachricht an das ganze Spiel weiterschickt.

