\chapter{Implementation von Beispielapps}
\label{implementation}
\section{Allgemein}
\subsection{Spiellogik}
\subsubsection{Server}
Der implementierte Server arbeitet mit Java und ZeroMQ. Er �ffnet zwei Kommunikationskan�le mit dem Client. Der erste ist ein ZeroMQ-Request-Reply-Socket-Paar, �ber das die Endger�te Nachrichten an den Server senden. Der zweite ist ein Publish-Subscribe-Socket-Paar, �ber das die Nachrichten an Gruppen von Endger�ten weitergeleitet werden. Eine Nachricht besteht aus drei Teilen: einer Adresse, dem Nachrichtentyp und einem serialisierten Objekt vom Typ TransferObject (siehe Abbildung \ref{fig:transfer}). Die Adresse ist entweder die ID eines Spielers oder die ID einer Spielinstanz. Wir nutzen ZeroMQs Multipart-Message-Feature um diese Teile voneinander getrennt bei der Kommunikation zu �bermitteln. Der Server sendet eingehende Nachrichten an bestimmte Clients weiter. Das kann entweder ein einzelner Client, oder alle Clients die sich in einer Spielsession befinden, sein. Der Server kennt dabei den Zustand der einzelnen Spiele nicht. , Um mehrere Spielinstanzen gleichzeitig verwalten zu k�nnen, h�lt er aber eine Liste aller laufenden Spiele samt Informationen dar�ber, welcher Spieler an welchem Spiel teilnimmt. Wenn eine Nachricht vom Typ \glqq create\_game\grqq empfangen wird tr�gt der Server die ID des Spiels in eine Liste ein. Diese Liste wird einem Client gesendet wenn er sie �ber eine entsprechende Nachricht anfragt.



\begin{figure}
	\begin{center}
		\includegraphics[width=1\textwidth]{img/snake.png}
	\end{center}
	
	\caption{UML-Diagramm zur Spiellogik}
	\label{fig:snake}
\end{figure}

\subsubsection{Client}
Unsere Android-Applikation besteht aus drei Fragmenten zwischen denen man durch Wischen wechseln kann. Das erste Fragment ist ein Chat über den Spieler Textnachrichten an ihre Mitspieler im gleichen Spiel senden können.
Das zweite Fragment ist die Karte auf der wir unser Spiel darstellen.  
Für die Darstellung der Karte verwenden wir GoogleMaps. 
Im dritten Fragment kann ein Spieler eine neue Spielinstanz erstellen oder einem bereits laufendem Spiel beitreten. Die Liste der momentan laufenden Spiele wird per Knopfdruck vom Server abgefragt. Wenn man einer laufenden Session beitritt wird über den Server eine Anfrage des aktuellen Status des Spiels an alle momentanen Mitspieler gesendet, welche den Zustand des Spiels alle an den anfragenden Client zurück senden. Dieser beachtet nur die erste Nachricht und verwirft den Rest.
Des weiteren läuft ein LocationClient, welcher immer die aktuelle Position als Nachricht an alle Spieler der selben Spielinstanz weiterschickt. 

\section{Der Android Client}

\subsection{Aufbau der GUI}

GUIIIIIIIIIIIIIIIIIIIIIIIIIIIIIIIIIIIIIIIIIIIIIIIIIIIIIIIIIIIIIIIIIIIIIIIIIIIIIIIIIIIIIIIIIIIIIIII
IIIIIIIIIIIIIIIIIIIIIIIIIIIIIIIIIIIIIIIIIIIIIIIIIIIIIIIIIIIIIIIIIIIIIIIIIIIIIIIIIIIIIIIIIIIIIIIIII
IIIIIIIIIIIIIIIIIIIIIIIIIIIIIIIIIIIIIIIIIIIIIIIIIIIIIIIIIIIIIIIIIIIIIIIIIIIIIIIIIIIIIIIIIIIIIIIIII

\subsection{Kommunikation}

blaaaaaaaaaaaaaaa aaaaaaaaaaaaa aaaaaaaaaaaaaaaaaaaaaaaaaaaaaaaaaaaaaaaaaaaaaaaaaa aaaaaaaaaaaaaa aaaaaaaaaaaaaaa aaaaaaaaa
blaaaaaaaaaaaaaaaaa aaaaaaaaaaaaaaaaaaaaa aaaaaaaaaaaaaaaaaaaaaaa aaaaaaaaaaaaaaaaaaaaaaaaaaaaaaaaaaaaaaaaaaaaaa aaaaaaaaaa
blaaaaaaaaaaaaaaaaaaaaaaaaaaaaaaaaaaaaaaaaaaaaaaaaaaaaaaaaaaaaaaaaaaaaaaaa aaaaaaaaaaaaaaaaa aaaaaaaaaaaaaaaaaaaaaaaaaaaa

\subsection{Die Karte}

\subsubsection{OpenStreetMap vs. GoogleMaps}

READY????! 3 2 1 FIGHT!!!!!!!!!!!!!!
\input{5-_Projektarchitektur/5-1_Der_Android_Client/5-1-3_Die_Karte/5-1-3-2_Fazit/OSM_vs_GMS_Fazit.tex}
\subsection{Die Karte}

\subsubsection{LocationManager vs. LocationClient}

READY????! 3 2 1 FIGHT!!!!!!!!!!!!!!
\subsubsection{Fazit}

AAAAAAND the WINNNNNNER ISSSSS ..........
\subsection{Spiellogik}

Lebe lange und in Frieden.

\section{Der ZeroMQ Server}

\subsection{Aufbau}

fadi dksfia adfasdfk dkflajaldflaee  aadferdfaerddadfetrdasdfressdr


\subsection{Kommunikation}

blaaaaaaaaaaaaaaa aaaaaaaaaaaaa aaaaaaaaaaaaaaaaaaaaaaaaaaaaaaaaaaaaaaaaaaaaaaaaaa aaaaaaaaaaaaaa aaaaaaaaaaaaaaa aaaaaaaaa
blaaaaaaaaaaaaaaaaa aaaaaaaaaaaaaaaaaaaaa aaaaaaaaaaaaaaaaaaaaaaa aaaaaaaaaaaaaaaaaaaaaaaaaaaaaaaaaaaaaaaaaaaaaa aaaaaaaaaa
blaaaaaaaaaaaaaaaaaaaaaaaaaaaaaaaaaaaaaaaaaaaaaaaaaaaaaaaaaaaaaaaaaaaaaaaa aaaaaaaaaaaaaaaaa aaaaaaaaaaaaaaaaaaaaaaaaaaaa


