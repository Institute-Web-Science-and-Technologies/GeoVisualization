\subsection{Spielidee}
Nachdem wir ein virtuelles Spiel in eine reale Umgebung integriert hatten, wollten wir uns Gedanken darüber machen, wie wir ein reales Spiel mit virtuellen Features erweitern könnten. 

Dabei fiel unsere Wahl auf ein einfaches Kinderspiel. Bei Variationen von Räuber und Gendarm könnte eine Seite Updates bekommen, wo oder in welcher Richtung die andere Seite gerade ist. Wenn der Gejagte Informationen über die Position des Jäger bekommen würde, wäre das Spiel deutlich dynamischer und man könnte mit mehr Jägern spielen. Wenn der Jäger wiederum eine Ahnung hätte, wo der Gejagte gerade ist, könnte man das Spielfeld deutlich größer machen. So könnte man aus einem Verstecken zum Beispiel eine Mister X- oder James Bond-Verfolgungsjagd in der Innenstadt machen und ein Kinderspiel zu etwas machen, das möglicherweise auch für Erwachsene noch interessant ist.

Dabei ist Ahnung das Schlüsselwort. Wenn man Angaben mit der Genauigkeit einer GPS-Angabe auf einer Karte bekommen würde, wäre das Spiel wenig spaßig, ein Versteck funktioniert halt nicht gut, wenn der Jäger genau weiß, an welcher Position der Gesuchte ist. Also muss man GPS Angaben entweder in ihrer Genauigkeit oder Verfügbarkeit stark beschränken. Man könnte z.B. nur alle paar Minuten ein kurzes Update der momentanen Position aller Spieler schicken, ähnlich wie das in Mister X der Fall ist. Statt auf einer Karte die Ausgabe darzustellen, könnte man auch das Handy vibrieren lassen (siehe 4.6 haptisches Feedback), wenn es in die Richtung eines Mitspielers zeigt. Oder man könnte es vibrieren lassen, wenn der Gesuchte oder der Jäger in der näheren Umgebung ist. Dazu ließe sich neben GPS auch Bluetooth nutzen (siehe 4.6 BlEep). Ein solches Feature ließe sich z.B. auch in "Blinde Kuh" umsetzen.

Allein dieses kurze Beispiel zeigt auf, wie leicht man reale Spiele erweitern kann und wie viele Möglichkeiten dahingehend existieren. Angesichts der enormen Verbreitung von Smartphones ist es durchaus vorstellbar, dass die Kinderspiele von morgen mit Smartphones gespielt werden, insbesondere da viele Eltern immer noch einen hohen Wert darauf legen, dass ihr Kind draußen spielt.