
\subsection*{Spielidee}
Das zweite Spiel, das von uns umgesetzt wurde, ist ein Flaggenspiel. Hierbei handelt es sich um eine Abwandlung von Capture the Flag. Es wird mit Stealth-Elementen erweitert und die Spielmechanik des Ausschaltens von Gegnern ist anders gel�st. Dieses Spiel sollte nach M�glichkeit in einer Stadt gespielt werden, da es wichtig ist in einer Menge von Passanten m�glichst unerkannt zu bleiben. Es gibt zwei Teams, die jeweils eine Basis haben. An den Standorten, wo die ersten Spieler den Teams beitreten, wird jeweils eine Basis erstellt. Die Zuweisung der Teams erfolgt gleichm��ig und ist zuf�llig. Die eigene Flagge erscheint in der N�he der eigenen Basis. Die Position der eigenen Flagge ist den Teammitgliedern unbekannt, man sieht nur die Position der gegnerischen Flagge. Die Flaggen sollen nur auf Stra�en bzw. Gehwegen (nicht in Geb�uden) zuf�llig platziert werden. Diese Platzierung wird vom Server �ber Bilderkennung bestimmt. So wird die Erreichbarkeit der Flaggen sichergestellt. Als Spieler sieht man nur die Positionen der eigenen Teammitglieder. Zur Organisation kann der Chat verwendet werden. Es ist m�glich die n�here Umgebung nach Gegnern zu scannen. Die gescannten Gegner sind dann f�r kurze zeit sichtbar. Dieses Scan-Verfahren ist jedoch nicht dauerhaft verf�gbar, sondern erst wieder m�glich nach l�ngerer Abklingzeit. Es ist m�glich potentielle Gegner zu markieren. Die Reichweite beim Markieren ist begrenzt und Richtungsabh�ngig. Hierbei wird das Smartphone auf den potentiellen Feind ausgerichtet und ein Mark-Button gedr�ckt (Auch hier besteht eine gewisse Abklingzeit). Ist dieser ein gegnerischer Mitspieler wird er markiert und ist f�r l�ngere Zeit auf der Karte sichtbar. Der Betroffene bekommt �ber seine Markierung keine R�ckmeldung. Ein Spieler, der sich in der N�he der gegnerischen Flagge befindet, kann diese aufnehmen. Um einen Punkt zu erreichen muss die aufgenommene Flagge zur eigenen Basis gebracht werden. Dem Gegnerteam wird die Position angezeigt, an der die Flagge aufgenommen wurde. Wird ein Flaggentr�ger markiert, verliert er die Flagge und sie erscheint an einer neuen zuf�lligen Position. War der Flaggentr�ger vor der Aufnahme markiert, so ist er mit Flagge zu sehen. Wenn ein Spieler eine bestimmte Geschwindigkeit �berschreitet, wird er ebenfalls f�r das Gegnerteam sichtbar. Somit haben schnellere L�ufer keinen direkten Vorteil.
