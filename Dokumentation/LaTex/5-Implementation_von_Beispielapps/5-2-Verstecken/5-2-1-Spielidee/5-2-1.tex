\section{Implementation von Beispielapps}
\subsection*{Spielidee}
Nachdem wir ein virtuelles Spiel in eine reale Umgebung integriert hatten, wollten wir uns Gedanken dar�ber machen, wie wir ein reales Spiel mit virtuellen Features erweitern k�nnten. 

Dabei fiel unsere Wahl auf ein einfaches Kinderspiel. Bei Variationen von R�uber und Gendarm k�nnte eine Seite Updates bekommen, wo oder in welcher Richtung die andere Seite gerade ist. Wenn der Gejagte Informationen �ber die Position des J�ger bekommen w�rde, w�re das Spiel deutlich dynamischer und man k�nnte mit mehr J�gern spielen. Wenn der J�ger wiederum eine Ahnung h�tte, wo der Gejagte gerade ist, k�nnte man das Spielfeld deutlich gr��er machen. So k�nnte man aus einem Verstecken zum Beispiel eine Mister X- oder James Bond-Verfolgungsjagd in der Innenstadt machen und ein Kinderspiel zu etwas machen, das m�glicherweise auch f�r Erwachsene noch interessant ist.

Dabei ist Ahnung das Schl�sselwort. Wenn man Angaben mit der Genauigkeit einer GPS-Angabe auf einer Karte bekommen w�rde, w�re das Spiel wenig spa�ig, ein Versteck funktioniert halt nicht gut, wenn der J�ger genau wei�, an welcher Position der Gesuchte ist. Also muss man GPS Angaben entweder in ihrer Genauigkeit oder Verf�gbarkeit stark beschr�nken. Man k�nnte z.B. nur alle paar Minuten ein kurzes Update der momentanen Position aller Spieler schicken, �hnlich wie das in Mister X der Fall ist. Statt auf einer Karte die Ausgabe darzustellen, k�nnte man auch das Handy vibrieren lassen (siehe 4.6 haptisches Feedback), wenn es in die Richtung eines Mitspielers zeigt. Oder man k�nnte es vibrieren lassen, wenn der Gesuchte oder der J�ger in der n�heren Umgebung ist. Dazu lie�e sich neben GPS auch Bluetooth nutzen (siehe 4.6 BlEep). Ein solches Feature lie�e sich z.B. auch in "Blinde Kuh" umsetzen.

Allein dieses kurze Beispiel zeigt auf, wie leicht man reale Spiele erweitern kann und wie viele M�glichkeiten dahingehend existieren. Angesichts der enormen Verbreitung von Smartphones ist es durchaus vorstellbar, dass die Kinderspiele von morgen mit Smartphones gespielt werden, insbesondere da viele Eltern immer noch einen hohen Wert darauf legen, dass ihr Kind drau�en spielt.