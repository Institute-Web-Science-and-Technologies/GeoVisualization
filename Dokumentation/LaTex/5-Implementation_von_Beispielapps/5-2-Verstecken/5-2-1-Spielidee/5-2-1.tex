\section{Implementation von Beispielapps}
\subsection*{Spielidee}
Als zweites Spiel, das wir umsetzen haben wir uns f�r ein Flaggenspiel entschieden. Hierbei handelt es sich um eine Abwandlung von Capture the Flag, welches mit Stealth-Elementen erweitert wird. Zudem wird hier die Spielmechanik des ausschalten von Gegnern anders gel�st. Dieses Spiel ist daf�r vorgesehen, dass es in St�dten gespielt wird, da es wichtig in einer Menge von Passanten m�glichst unerkannt zu bleiben. Es gibt zwei Teams, jeweils mit einer Basis. Der Standpunkt das Basis wird jeweils am Standort des Spieler der als erstes dem Team beitritt erstellt. Die Zuweisung der Teams erfolgt gleichm��ig und zuf�llig. Die eigene Flagge erscheint in der n�he eigenen Basis, deren Position ist den Teammitgliedern unbekannt, man sieht nur die Position der gegnerischen Flagge. Die "Spawn-Positionen" der Flaggen sollen nur Stra�en sein. Diese werden vom Server �ber Bilderkennung bestimmt. So wird die Erreichbarkeit der Flaggen sichergestellt. Als Spieler sieht man nur die Position des eigenen Teams. Zur Organisation kann der Chat verwendet werden. Es ist m�glich die nahe Umgebung nach Gegnern zu scannen(erst wieder m�glich nach l�ngerer Abklingzeit), diese sind dann f�r kurze zeit Sichtbar. Es ist m�glich potentielle Gegner zu markieren. Die Reichweite es Markieren ist begrenzt und Richtungsabh�ngig. Hierbei wird das Smartphone auf den potentiellen Feind ausgerichtet und den Mark-Button gedr�ckt(k�rzere Abklingzeit). Ist dieser ein gegnerischer Mitspieler wird er markiert und ist f�r l�ngere Zeit sichtbar. Der betroffene bekommt keine R�ckmeldung, das er markiert ist. Wenn man sich in der n�he der gegnerischen Flagge befindet, kann man diese aufnehmen. Diese gilt es dann zur eigenen Basis zu bringen und einen Punkt zu erzielen. Dem Gegnerteam wird das Aufnehmen der Flagge mit Position angezeigt. Wird der Flaggentr�ger dann markiert, verliert er die Flagge und diese erscheint an einer neuen Position. War der Flaggentr�ger vor der Aufnahme markiert, so ist er mit Flagge zu sehen. Wenn ein Spieler eine bestimmte Geschwindigkeit �ber wird er auch f�r das Gegnerteam sichtbar, somit haben schnellere L�ufer keinen direkten Vorteil.
