\subsection{Spiellogik}

\subsubsection*{Server}
Der Server wird hier noch mit einer Zusatzfunktion ausgestattet. Er generiert nach vorher festgelegten Bedingung g�ltige Flaggenpunkte. In diesem Fall z�hlen Stra�en als g�ltige Punkte. Der Server ruft �ber GoogleStaticMap einen Kartenabschnitt ab, auf dem der Flaggenpunkt generiert werden soll. Nun wird ein zuf�lliger Pixel in der Mitte des Bildes nach der Farbe �berpr�ft. Stimmt dieser mit der Farbe von Stra�en �berein, wird diese Position als g�ltig zur�ck �bertragen und der Flaggenpunkt kann generiert werden. Ist die Farbe ung�ltig wird ein entsprechender neuer Kartenabschnitt aufgerufen und die Prozedur beginnt erneut.

\subsubsection*{Client}
Auf der Client-Seite wird die Sichtbarkeit der Spieler und Objekte (Flaggen und Basen) gehandhabt.

\paragraph{Markieren}
 S�mtliche Positionen sind den Gerten bekannt, aber nicht alle sichtbar. Damit verbunden sind auch die Funktionalit�ten des Markieren und Scannens. F�r das Markieren werden jeweils die beiden Koordinaten des Markierenden (A) und des Markierten (B), sowie den Sichtkegel von A und eine Reichweite ben�tigt. 
F?r den Sichtkegel ben�tigt man einen Orientierungswinkel ($\alpha$), welche mittels der Sensoren f�r Orientierung (s. \ref{orientierung}) ermittelt wird, sowie einen vorher festgelegten Winkel f�r den Sichtbereich ($\beta$). Nun wird der Winkel der beiden Punkte A und B ($\gamma$) bestimmt (s. \ref{abstandsmessung}).


Liegt $\gamma$ im Intervall $[ \alpha - \frac{\beta}{2}; \alpha + \frac{\beta}{2}]$ und ist die Distanz zwischen A und B innerhalb der Reichweite, so gilt B als markiert. �ber eine Nachricht wird den anderen Teammitgliedern mitgeteilt das B jetzt markiert und somit f�r eine gewisse Zeit sichtbar ist. A kann f�r eine vordefinierte Zeit keine weiteren Gegner markieren.

\paragraph{Scannen}

Um einen festgelegten Radius eines Spielers nach Gegnern zu scannen, wird die Distanz (s. \ref{abstandsmessung}) zwischen dem Scannenden (C) und allen gegnerischen Spielern ermittelt. C werden, f�r eine kurze Zeit, alle gegnerischen Spieler angezeigt, die sich innerhalb eines bestimmten Distanzwertes befinden. Die M�glichkeit zu Scannen steht C f�r eine gewisse Zeit nicht mehr zur Verf�gung.

\paragraph{Geschwindigkeits�berpr�fung}
Es existiert eine st�ndig aktive Geschwindigkeits�berpr�fung der Spieler durch ihre Koordinaten (s. \ref{locationManager}). Wenn eine feste Maximalgeschwindigkeit durch einen Spieler �berschritten wird, schickt der entsprechende Client eine Nachricht an alle gegnerischen Ger�te, dass dieser angezeigt werden soll. Wird die Maximalgeschwindigkeit wieder unterschritten, ist der Betroffene wieder unsichtbar f�r die Gegner

\paragraph{Basen- und Flaggenplatzierung}
Die Basis eines Teams wird vom ersten Teammitglied gesetzt, welches auf den "`set Base"' Button klickt. Als Zentrum wird die momentane Position des Spielers genommen. Die Darstellung erfolgt als Kreis. Nun wird in einem vorher festgelegtem Radius um die Basis zuf�llig die Teamflagge platziert. Die Auswahl f�r eine g�ltige Koordinate trifft dabei der Server und teilt dies den entsprechenden Mitspielern �ber eine Nachricht mit. Die aktuelle Position der Flagge wird jedoch dem eigenen Team nicht angezeigt. Wird die Flagge von einem Mitspieler des gegnerischen Team gestohlen, werden die betroffenen Spieler �ber einen Vibrationsalarm informiert. Auch wird die letzte Position der Flagge vorl�ufig angezeigt. Wenn der Flaggentr�ger gestoppt wird, generiert der Server wieder eine g�ltige Position f�r eine neue Flagge 