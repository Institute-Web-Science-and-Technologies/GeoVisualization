\subsection{Spiellogik}

\subsubsection*{Server}


\subsubsection*{Client}
Auf der Client-Seite wird die Sichtbarkeit der Spieler und Objekten (Flaggen und Basen) gehandhabt. Sämtliche Positionen sind den Geräten bekannt, werden aber je nach dem vor dem Spieler versteckt. Damit verbunden sind auch die Funktionalitäten des Markieren und Scannen. Für das Markieren werden jeweils die beiden Koordinaten des Markierenden (A) und des Markierten (B), sowie den Sichtkegel von A und eine Reichweite benötigt. 
Für den Sichtkegel benötigt man einen Orientierungswinkel ($\alpha$), welche mittels der Sensoren für Orientierung (s. MUSS noch geschrieben werden) ermittelt wird, sowie einen vorher festgelegten Winkel für den Sichtbereich ($\beta$). Nun wird der Winkel der beiden Punkte A und B ($\gamma$) bestimmt (s. TEXT ÜBER ORTHODROME DER NOCH GESCHRIEBEN WERDEN MUSS).

Liegt $\gamma$ im Intervall $[ \alpha - \frac{\beta}{2}; \alpha + \frac{\beta}{2}]$ und ist die Distanz zwischen A und B innerhalb der Reichweite, so gilt B als markiert. Über eine Nachricht wird den anderen Teammitglieder mitgeteilt das B jetzt markiert und somit für eine gewisse Zeit Sichtbar ist. A kann für eine gewisse Zeit keine weiteren Gegner markieren.

Um einen vorher festgelegten Radius um den Spieler nach Gegnern zu scannen, wird die Distanz (s. TEXT ÜBER ORTHODROME) zwischen dem Scannenden (C) und allen gegnerischen Spieler gemessen. C werden, für eine kurze Zeit alle gegnerischen Spieler angezeigt, die sich innerhalb eines vorher festgelegten Distanzwertes befinden. Die Möglichkeit zu Scannen steht dem auslösenden Spieler für eine gewisse Zeit nicht mehr zur Verfügung.

Eine ständig aktive Geschwindigkeitsüberprüfung der Koordinaten (s. \ref{locationManager})  stellt sicher, das Spieler sich an die vorher festgelegte Maximalgeschwindigkeit halten. Wenn eine Überschreitung eintritt, schickt der entsprechende Client eine Nachricht an alle gegnerischen Geräte, dass dieser angezeigt werden soll. Wird die Maximalgeschwindigkeit wieder unterschritten, ist der betroffene Spieler wieder unsichtbar für die Gegner.




