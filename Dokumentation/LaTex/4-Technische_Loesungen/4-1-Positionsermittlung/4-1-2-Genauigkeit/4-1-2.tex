\subsection{Evaluation der Genauigkeit der Positionsermittlung mit unterschiedlichen Diensten}

  \begin{figure}[h]
    \begin{center}
    \includegraphics[width=1.0\textwidth]{4-Technische_Loesungen/4-1-Positionsermittlung/Data/updates_per_second_bigger.png}
    \end{center}
     \caption{Updates pro Sekunde}
     \label{fig: picUdates}
  \end{figure}
	
	\begin{figure}[h]
    \begin{center}
    \includegraphics[width=1.0\textwidth]{4-Technische_Loesungen/4-1-Positionsermittlung/Data/accuracy_bigger.png}
    \end{center}
     \caption{Genauigkeit}
     \label{fig: picAccu}
  \end{figure}

Um LocationManager und LocationClient f�r unsere Zwecke vergleichen zu k�nnen, wurde ein Testprogramm entwickelt, welches jeweils in zwei separaten Activities GPS-Messdaten in einer Datei sammelt. Je eine Activity verwendet dabei den LocationManager und den LocationClient, um in regelm��igen Abst�nden Koordinaten und deren approximierte Genauigkeit abzurufen und abzuspeichern. Das Experiment wird durchgef�hrt, indem eine Person mit dem Ger�t, auf dem das Testprogramm ausgef�hrt wird, eine m�glichst gerade Strecke abgeht. Zur optischen Darstellung der Ergebnisse wird aus den aufgezeichneten Koordinaten eine Linie (PolyLine\footnote{\url{https://developer.android.com/reference/com/google/android/gms/maps/model/Polyline.html}}) gezeichnet. Als Vergleich wird eine Luftlinie zwischen Anfangs- und Endpunkt gezogen. Die ausgewerteten Sensordaten werden in Anhang \ref{app} dargestellt.

Der Genauigkeitswert ist hierbei folgenderma�en definiert: Um die ermittelte Position wird ein Kreis mit der ermittelten Genauigkeit als Radius gezeichnet. Der verwendete Dienst sch�tzt nun, dass
sich die tats�chliche Position des Ger�ts mit einer 68-prozentigen Wahrscheinlichkeit innerhalb dieses Kreises um die von dem Dienst ermittelte Position befindet\footnote{\url{http://developer.android.com/reference/android/location/Location.html}}.
Es gilt, je kleiner der Wert desto besser. In den Tabellen in Anhang \ref{app} wurde jeweils das arithmetische Mittel aller im Zuge des entsprechenden Experiments aufgezeichneten Genauigkeiten eingetragen.
	  

Die Auswertung der Daten hat ergeben, dass bez�glich der Genauigkeit die vom LocationManager ermittelten Daten in sechs von elf Versuchen genauer als die vom LocationClient ermittelten sind (siehe Graph \ref{fig: picAccu}). Im Bezug auf Updates pro Sekunde liefert der LocationClient in neun von elf Versuchen die besseren Ergebnisse (siehe Graph \ref{fig: picUdates}).

Mit mobilen Endger�ten erzeugte Positionsdaten haben meist eine Genauigkeit von 2 bis 15 Metern, dabei sind die Ergebnisse in offenem Gel�nde besser, als in engen Gassen \cite{gpsacc}. Befindet sich das Ger�t innerhalb eines Geb�udes sind die Ergebnisse unbrauchbar. Beides wurde durch die im Zuge der Evaluation durchgef�hrten Experimente best�tigt. Zus�tzlich hat auch die Witterung Einfluss auf die Genauigkeit der GPS-Positionen. Da beim Testen der entwickelten Anwendungen Ungenauigkeiten von f�nf oder mehr Metern als extrem st�rend empfunden wurden, k�nnen solche Spiele nur auf einem gen�gend gro�en, offenen Gel�nde gespielt werden. Die Updategeschwindigkeit von etwa einer Sekunde hat sich als ausreichend erwiesen, falls sich alle Mitspieler in Schrittgeschwindigkeit bewegen.

%\subsection{Fazit}
Genauigkeit ist in diesem Projekt ein wichtiger Faktor, vor allem im Bezug auf Kollisionen mit anderen Spielern und virtuellen Objekten. Da in diesem Projekt kompetitive Mehrspieler-Spiele realisiert werden, ist auch die Updategeschwindigkeit zu ber�cksichtigen. 
%Die Versuche haben gezeigt, dass bez�glich der Genauigkeit zwischen den beiden Dienste kein statistisch signifikanter Unterschied festgestellt werden kann. 
Da die im Zuge dieses Projekts entwickelten Anwendungen zur Umgebungsdarstellung auf GoogleMaps zu\-r�ck\-grei\-fen (siehe Abschnitt \ref{kartendarstellung}), das sowohl Google-Play-Services, als auch eine konstante Internetverbindung ben�tigt, kann der LocationClient ohne Bedenken wegen dieser Abh�ngigkeiten und fehlender Abw�rtskompabilit�t eingesetzt werden.
Da dieser zus�tzlich durch tendenziell geringere Zeitintervalle zwischen den gelieferten Werten einen fl�ssigeren Spielablauf erm�glicht, wird in diesem Projekt der LocationClient verwendet. 


 