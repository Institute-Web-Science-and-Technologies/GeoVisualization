\subsection{Genauigkeit}
Nun folgen eine grafische Darstellung der Genauigkeitsaufzeinungen die mit dem Testprogramm gemacht wurden

  \begin{figure}[h]
    \begin{center}
    \includegraphics[width=0.48\textwidth]{4-Technische_Loesungen/4-1-Positionsermittlung/Data/Screenshot_2014-12-10-12-41-12_fuusion.png}
    \end{center}
     \caption{Location Versuch 1}
     \label{fig: picLV1}
  \end{figure}

  \begin{figure}[h]
    \begin{center}
    \includegraphics[width=0.48\textwidth]{4-Technische_Loesungen/4-1-Positionsermittlung/Data/Screenshot_2014-12-14-15-33-10_fuusion.png}
     \end{center}
     \caption{Location Versuch 2}
     \label{fig: picLV2}
  \end{figure}
  
  \begin{figure}[h]
    \begin{center}
    \includegraphics[width=0.48\textwidth]{4-Technische_Loesungen/4-1-Positionsermittlung/Data/Screenshot_2015-05-08-16-09-35_fuusion.png}
     \end{center}
     \caption{Location Versuch 3}
     \label{fig: picLV3}
  \end{figure}
	
	\begin{figure}[h]
    \begin{center}
    \includegraphics[width=0.48\textwidth]{4-Technische_Loesungen/4-1-Positionsermittlung/Data/Screenshot_2015-05-08-16-37-21_fuusion.png}
     \end{center}
     \caption{Location Versuch 4}
     \label{fig: picLV4}
  \end{figure}
	
	\begin{figure}[h]
    \begin{center}
    \includegraphics[width=0.48\textwidth]{4-Technische_Loesungen/4-1-Positionsermittlung/Data/Screenshot_2015-05-08-17-03-10_fuusion.png}
     \end{center}
     \caption{Location Versuch 5}
     \label{fig: picLV5}
  \end{figure}
  
  \begin{figure}[h]
    \begin{center}
    \includegraphics[width=0.48\textwidth]{4-Technische_Loesungen/4-1-Positionsermittlung/Data/2015-05-08_16-14-46_fuusion.png}
     \end{center}
     \caption{Location Versuch 6}
     \label{fig: picLV6}
  \end{figure}
  
  \begin{figure}[h]
    \begin{center}
    \includegraphics[width=0.48\textwidth]{4-Technische_Loesungen/4-1-Positionsermittlung/Data/2015-05-08_17-01-38_fuusion.png}
     \end{center}
     \caption{Location Versuch 7}
     \label{fig: picLV7}
  \end{figure}
  
  \begin{figure}[h]
    \begin{center}
    \includegraphics[width=0.48\textwidth]{4-Technische_Loesungen/4-1-Positionsermittlung/Data/Screenshot_2015-05-08-17-07-04_fuusion.png}
     \end{center}
     \caption{Location Versuch 8}
     \label{fig: picLV8}
  \end{figure}
  
  \begin{figure}[h]
    \begin{center}
    \includegraphics[width=0.48\textwidth]{4-Technische_Loesungen/4-1-Positionsermittlung/Data/Screenshot_2015-05-08-17-24-35_fuusion.png}
     \end{center}
     \caption{Location Versuch 9}
     \label{fig: picLV9}
  \end{figure}
  
  \begin{figure}[h]
    \begin{center}
    \includegraphics[width=0.48\textwidth]{4-Technische_Loesungen/4-1-Positionsermittlung/Data/Screenshot_2015-05-08-17-39-18_fuusion.png}
     \end{center}
     \caption{Location Versuch 10}
     \label{fig: picLV10}
  \end{figure}
  
  \begin{figure}[h]
    \begin{center}
    \includegraphics[width=0.48\textwidth]{4-Technische_Loesungen/4-1-Positionsermittlung/Data/2015-05-08_17-34-52_fuusion.png}
     \end{center}
     \caption{Location Versuch 11}
     \label{fig: picLV11}
  \end{figure}
  
\clearpage

\subsection{Fazit}
Die Messdaten zeigen, dass beide Dienste auf unterschiedlichen Ger�ten zu unterschiedlichen Ergebnissen f�hren. Ein klarer Favorit ist nicht auszumachen. Bez�glich Updates pro Sekunde liefert der LocationClient minimal die besseren Ergebnisse. Ein weiteres Faktor ist, dass sich schon bereits f�r die Kartenanzeige f�r GoogleMaps entschieden wurde. Es ist davon auszugehen,dass GoogleMaps und LocationClient besser harmonieren bzw. Kompatibilit�tsprobleme weniger wahrscheinlich sind, da beides von Google Play Services bereitgestellt werden. In diesem Projekt wird daher der LocationManager verwendet.