\subsection{LocationManager vs LocationClient}

Im folgenden werden zwei m�gliche Dienste zur Positionsermittlung verwendet. Der LocationManager und der Location client.


\subsubsection{LocationManager}

Der LocationManager ist ein von Android breitgestellter Dienst, der �ber den GPS Sensor die Position des Smartphones ermittelt.
Ein Vorteil bei verwendenden dieses Dienstes ist, das man nicht an die Dienste von Google gebunden ist, und somit theoretisch auch f�r
nicht Google-Konforme Androidbetriebssysteme entwickeln kann. Da nur der GPS Sensor verwendet wird, w�re eine entprechende Anwendung sogar unabh�ngig von einer Internetverbindung. Auch wird dieser Dienst auch noch von �lteren Android Versionen unterst�tzt.
Nachteilig wird in einer Forendiskussion \footnote{http://stackoverflow.com/questions/20908822/android-difference-between-locationmanager-addproximityalert-locationclien} der Hohe Akku verbrauch genannt.


\subsubsection{LocationClient}

Der LocationClient ist ein von Google bereitgestellter Dienst, der �ber den GPS Sensor und andere Quellen, die Google zur Verf�gung stellt, die Position des Smartphones zu ermitteln. Als Vorteile dieses Dienstes werden in der oben schon erw�hnten Forendiskussion wird erw�hnt, dass dieser Dienst akkuschonender sei. Auch sei er zuvel�ssiger. Die anderen Quellen zur Positionsermittlung k�nnten auch zu einer genauen Positionsermittlung beitragen, wenn es eventuell �bertragungsprobleme zum GPS-Satelliten gibt.
Nachteilig an diesem Dienst ist, dass man eine Internetverbindung ben�tig. Au�erdem werden die Google Play Services vorausgesetzt, welche erst ab Android Version 2.2 zur Verf�gung stehen. Zudem ist man auch total von Google abh�ngig.


\subsubsection{Messdaten}

Um LocationManager und LocationClient f�r unsere zwecke vergleichen zu k�nnen, wurde ein kleines Testprogramm entwickelt, welches jeweils in zwei seperaten Activities GPS Messdaten in einer Datei sammelt. Es werden jeweils Koordinaten und Genauigkeit gesammtelt. Zur optischen Darstellung der Ergenisse wird aus den aufgezeichneten Koordinate eine Linie (PolyLine) gezeichet. Als vergleich wird eine Luftlinie zwischen Anfangs- und Endpunkt gezogen. \newline
Im Folgenden werden nun die ausgewerteten Sensordaten dargestellt.

\begin{table}[h]
\caption{Location Versuch 1}
\begin{tabular}{l|l|l|}
\cline{2-3}
                                      & locationClient         & locationManager         \\ \hline
\multicolumn{1}{|l|}{Datum}           & \multicolumn{2}{l|}{10.12.2014}                  \\ \hline
\multicolumn{1}{|l|}{Ort}             & \multicolumn{2}{l|}{Universit�t Koblenz, Campus} \\ \hline
\multicolumn{1}{|l|}{Wetter}          & \multicolumn{2}{l|}{Bew�lk, leichter Niesel}     \\ \hline
\multicolumn{1}{|l|}{Ger�t}           & \multicolumn{2}{l|}{Samsung Galaxy S3}           \\ \hline
\multicolumn{1}{|l|}{Genauigkeit}     & 7,049                  & 8,698                   \\ \hline
\multicolumn{1}{|l|}{Updates/Sekunde} & 0,936                  & 1,016                   \\ \hline
\end{tabular}
\label{tab:lV1}
\end{table}

\begin{table}[h]
\caption{Location Versuch 2}
\begin{tabular}{l|l|l|}
\cline{2-3}
                                      & locationClient                & locationManager                \\ \hline
\multicolumn{1}{|l|}{Datum}           & \multicolumn{2}{l|}{14.12.2014}                                \\ \hline
\multicolumn{1}{|l|}{Ort}             & \multicolumn{2}{l|}{Ransbach-Baumbach, L307richtung Mogendorf} \\ \hline
\multicolumn{1}{|l|}{Wetter}          & \multicolumn{2}{l|}{Wetter:,leicht bew�lk leicht windig}       \\ \hline
\multicolumn{1}{|l|}{Ger�t}           & \multicolumn{2}{l|}{Samsung Galaxy S3}                         \\ \hline
\multicolumn{1}{|l|}{Genauigkeit}     & 5,482                         & 6,216                          \\ \hline
\multicolumn{1}{|l|}{Updates/Sekunde} & 1,003                         & 0,969                          \\ \hline
\end{tabular}
\label{tab:lV2}
\end{table}

Hierbei muss noch erg�nzt werden, dass sich der Genaukeitswert folgenderma�en beschrieben wird. An L�ngen- und Breitengrad
der Position wird ein Kreis mit Radius der Genauigkeit gezeichnet. Es besteht eine 68-prozentige Wahrscheinlichkeit, dass
sich die Position innerhalb des Kreises Befindet\footnote{http://developer.android.com/reference/android/location/Location.html}.
Es gilt, je kleiner der Wert desto besser. In den Tabellen wurde jeweils das aritmetische Mittel der aufgezeichneten Genauigkeiten eingetragen.


