\subsection{Dienste zur Positionsermittlung}

Im folgenden werden zwei m�gliche Dienste zur Positionsermittlung verglichen. Der LocationManager\footnote{{\url{http://developer.android.com/reference/android/location/LocationManager.html}}} und der LocationClient\footnote{{\url{http://www.doc.ic.ac.uk/project/2013/271/g1327125/android-studio/sdk/extras/google/google\_play\_services/docs/reference/com/google/android/gms/location/LocationClient.html}}}.


\subsubsection{LocationManager}

Der LocationManager ist ein von Android bereitgestellter Dienst, der �ber den GPS Sensor die Position des Smartphones ermittelt, au�erdem werden die Positionen mit Geschwindigkeitswerten versehen.
Ein Vorteil dieses Dienstes ist, dass er im Gegensatz zum LocationClient nicht an von Google bereitgestellte Standardbibliotheken, wie beispielsweise Google-Play-Services, gebunden ist. Der LocationManager kann daher problemlos auf vielen Android-Ger�ten verwendet werden, auf denen die Installation der f�r den LocationClient ben�tigten Bibliotheken unm�glich oder zu aufwendig ist.
Da der LocationManager zur Positionsbestimmung ausschlie�lich vom GPS-Sensor bereitgestellte Daten verwendet, funktioniert er ohne Internetverbindung. Er ist �lter als der LocationClient und wird somit auch von �lteren Android-Versionen unterst�tzt.
%Nachteilig wird in einer Forendiskussion \footnote{\url{http://stackoverflow.com/questions/20908822/android-difference-between-locationmanager-addproximityalert-locationclien }} der hohe Akku verbrauch genannt.
Der LocationManager scheint den Akku in h�herem Ma�e zu belasten, als der LocationClient.
Die entwickelten Anwendungen werden aber nur kurzfristig f�r einzelne Spiele verwendet. Deshalb halten wir dies nicht f�r ausschlaggebend.

\subsubsection{LocationClient}

Der LocationClient ist ein in den Google-Play-Services enthaltener Dienst, der �ber den Dienst des LocationManager und anderen Quellen, wie beispielsweise WLAN-Netze in der n�heren Umgebung, die Position des Smartphones ermittelt. 
%Ein weiterer Vorteil dieses Dienstes ist - wie auch oben bereits erw�hnt - , das dieser Dienst Akku-schonender und zuverl�ssiger ist. 
Die weiteren Quellen zur Positionsermittlung k�nnen zu einer genauen Positionsermittlung beitragen, wenn es �bertragungsprobleme zum GPS-Satelliten gibt.
Nachteilig an diesem Dienst ist, dass der Endnutzer einen g�ltigen Google-Konto ben�tigt, welches f�r das verwenden von Google-Play-Services vorausgesetzt wird. Zudem stehen die Google-Play-Services erst ab Android Version 2.2 zur Verf�gung.





