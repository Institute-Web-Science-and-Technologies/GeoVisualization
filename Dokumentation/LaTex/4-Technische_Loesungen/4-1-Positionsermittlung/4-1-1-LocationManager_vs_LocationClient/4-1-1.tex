\subsection{LocationManager vs LocationClient}

Im folgenden werden zwei m�gliche Dienste zur Positionsermittlung verwendet. Der LocationManager und der Location client.


\subsubsection{LocationManager}

Der LocationManager ist ein von Android breitgestellter Dienst, der �ber den GPS Sensor die Position des Smartphones ermittelt.
Ein vorteil bei verwendenden dieses Dienstes ist, das man nicht an die Dienste von Google gebunden ist, und somit theoretisch auch f�r
nicht Google-Konforme Androidbetriebssysteme entwickeln kann. Da nur der GPS Sensor verwendet wird, w�re eine entprechende Anwendung sogar unabh�ngig von einer Internetverbindung. Auch wird dieser Dienst auch noch von �lteren Android Versionen unterst�tzt.
Nachteilig wird in diversen Foren (stackoverflow) der Hohe Akku verbrauch genannt.


\subsubsection{LocationClient}

Der LocationClient ist ein von Google bereitgestellter Dienst, der �ber den GPS Sensor und andere Quellen, die Google zur Verf�gung stellt, die Position des Smartphones zu ermitteln. Als Vorteile dieses Dienstes werden in Forendiskussionen unter anderem genannt, dass dieser Dienst
akkuschonender sei. Auch sei er zuvel�ssiger. Die anderen Quellen zur Positionsermittlung k�nnten auch zu einer genauen Positionsermittlung beitragen, wenn es eventuell �bertragungsprobleme zum GPS-Satelliten gibt.
Nachteilig an diesem Dienst ist, dass man eine Internetverbindung ben�tig. Au�erdem werden die Google Play Services vorausgesetzt, welche erst ab Android Version 2.2 zur Verf�gung stehen. Zudem ist man auch total von Google abh�ngig.


\subsubsection{Auswertung von Messdaten}

Um LocationManager und LocationClient f�r unsere zwecke vergleichen zu k�nnen, wurde ein kleines Testprogramm entwickelt, welches jeweils in zwei seperaten Activities GPS Messdaten in einer Datei sammelt. Es werden jeweils Koordinaten und genauigkeit Gesammtelt. Es ist jeweils eine GoogleMap implementiert, mit Buttons zum Starten und Stoppen der Datenaufzeichung. Ein Draw-Button l�sst zwei Linien zeichen. Eine direkte Luftlinie zwischen dem Start und dem Endpunkt und eine PolyLine aus dem empfangen Daten.
