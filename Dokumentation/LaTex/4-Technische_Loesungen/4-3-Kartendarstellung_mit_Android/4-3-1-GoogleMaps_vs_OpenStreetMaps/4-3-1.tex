%\subsection{GoogleMaps vs OpenStreetMaps}
Die Basis der Spielkonzepte ist eine Karte der reellen Umgebung, die mit Android dargestellt wird. Realisiert wurde dies mit Google Maps, obwohl Open Street Maps auch eine Alternative gewesen w�re. Im folgenden m�chten wir beide Optionen durchleuchten und die Vor- und Nachteile sowie die Unterschiede betrachten. 
Google Maps ist Eigent�mer aller Karten und Informationen die diese beinhalten. OSM dagegen kommt mit einer offenen Lizenz daher. Die zugrundeliegenden Geodaten der Karten stellt Google Maps nicht zur Verf�gung, OSM hingegen schon. Beide erhalten t�gliche Updates, wobei jedoch die Satellitenkarte von Google Maps nur alle 1 bis 3 Jahre aktuell gehalten wird. Diese wird von OSM allerdings nicht angeboten.
OSM bietet volle Funktionalit�t in allen L�ndern, wogegen diese bei Google Maps auf spezielle L�nder beschr�nkt ist. 
Als Datentyp im Backend nutzt Google Maps JSON, OSM setzt hingegen auf XML. Beide bieten Bibliotheken f�r Java und Javascript. Wobei die letzteren f�r OSM von Drittanbietern bereitgestellt werden. 
Beide nutzen GPS und Wi-Fi des Mobilen Endger�tes f�r die Positionsbestimmung und bieten die Darstellung der Bewegungsrichtung. Google Maps unterst�tzt zus�tzlich den Zugriff auf Mobilfunkzellen f�r die Ortsbestimmung.
Wir haben uns f�r Google Maps entschieden, weil die API sehr gut und zentral dokumentiert ist. Die Integration in die Android App war unkompliziert und die Karten-Funktionalit�ten waren leicht zu nutzen. Google bietet hier f�r alle Zwecke ausf�hrliche Tutorials an. Auch die Geodaten ben�tigten wir nicht. Einzig die Anzeige der Karte und die Interaktion der Benutzer untereinander war uns wichtig.
