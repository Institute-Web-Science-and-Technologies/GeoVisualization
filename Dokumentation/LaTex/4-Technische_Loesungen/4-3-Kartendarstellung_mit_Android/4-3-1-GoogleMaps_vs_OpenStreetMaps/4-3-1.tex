%\subsection{GoogleMaps vs OpenStreetMaps}
Die Basis aller entwickelten Spielkonzepte ist eine Karte der reellen Umgebung, die mit Android dargestellt wird. Realisiert wurde dies mit Google Maps. Eine alternative Technologie stellt OpenStreetMap (OSM) dar. Im folgenden Abschnitt werden beide Technologien verglichen und diese Entscheidung begr�ndet. 

Google ist Eigent�mer aller von Google Maps verwendeten Informationen und der dargestellten Karten. OSM dagegen ist unter einer offenen Lizenz ver�ffentlicht. Die zugrundeliegenden Geodaten der Karten stellt Google Maps nicht zur Verf�gung, w�hrend OSM diese offen zug�nglich macht. Beide erhalten t�gliche Updates, wobei jedoch die Satellitenkarte von Google Maps nur alle ein bis drei Jahre aktuell gehalten wird. Gleichzeitig stellt OSM allerdings keine Satellitenkarte bereit.

OSM bietet volle Funktionalit�t in allen L�ndern, wogegen diese bei Google Maps auf spezielle L�nder beschr�nkt ist. 
Als Datentyp im Backend nutzt Google Maps JSON\footnote{{\url{http://json.org/}}}, OSM setzt hingegen auf XML\footnote{{\url{http://www.w3.org/TR/2006/REC-xml11-20060816/}}}. Beide bieten Bibliotheken f�r Java und Javascript. Wobei die letzteren f�r OSM von Drittanbietern bereitgestellt werden. 

%Beide nutzen GPS und Wi-Fi des Mobilen Endger�tes f�r die Positionsbestimmung und bieten die Darstellung der Bewegungsrichtung. Google Maps unterst�tzt zus�tzlich den Zugriff auf Mobilfunkzellen f�r die Ortsbestimmung.
Die Datenbank von OSM wird von Freiwilligen gepflegt. Diese haben in der Vergangenheit vor allem Daten aus Ballungszentren geliefert, daher sind die Daten von OSM von regional von unterschiedlicher Qualit�t. Schlechte Ergebnisse werden vor allem im l�ndlichen Raum gemessen \cite{neis}.
Wir haben uns f�r Google Maps entschieden, weil die API sehr gut und zentral dokumentiert ist. Die Integration in die Android-App war unkompliziert und die Karten-Funktionalit�ten waren leicht zu nutzen. Google bietet hier f�r alle Zwecke ausf�hrliche Tutorials an. Auch die Geodaten ben�tigten wir nicht. Einzig die Anzeige der Karte und die Interaktion der Benutzer untereinander war uns wichtig.
