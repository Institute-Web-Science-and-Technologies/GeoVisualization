\section{Andere Sensorik}\label{sensorik}

\subsection*{BLEep}
BLEep bezeichnet eine Technologie, die Bluetooth Sender nutzt um Signale mit anderen Ger�ten auszutauschen. BLE (auch bekannt als Bluetooth Low Energy) sendet auf einer Frequenz von 2,4 GHz und nutzt Bluetooth Version 4.0, welche auch heute von allen g�ngigen Smartphones unterst�tzt wird.
Ein BLEep Sender wird an einem Ort angebracht und sendet dauerhaft Signale an alle umliegenden Ger�te. Die Reichweite des Sender liegt zwischen 15cm und 50m.
Mobile Ger�te empfangen das Signal und k�nnen an der Signalst�rke die Entfernung zum Sensor bestimmen. Je schw�cher das Signal, desto weiter ist die Entfernung.
Jeder BLEep Sensor besitzt eine ID, womit dieser sich eindeutig identifizieren l�sst.
Eine mobile Applikation empf�ngt also das Signal mit der ID und berechnet anhand der Signalst�rke die Entfernung. Aus diesem Kontext heraus k�nnen dann Interaktionen zwischen der mobilen App und dem Sensor durchgef�hrt werden. Dies ist auch mit mehreren Smartphones m�glich, denn die Sensoren haben keine Begrenzung hinsichtlich der Anzahl der Ger�te mit denen sie kommunizieren k�nnen. Auch kann ein Ger�t mit mehreren Sensoren gleichzeitig interagieren, was Spielraum f�r weitere m�gliche Szenarios l�sst.

\subsection*{Kompass}
Heute besitzen alle g�ngigen Smartphones eine Kompass Funktionalit�t, die von mobilen Applikationen genutzt werden kann. Doch wie funktioniert das?
Ganz allgemein richtet sich ein Kompass nach dem Magnetfeld der Erde. Er ist mit einer magnetischen Nadel ausgestattet, die sich auf den Nordpol einstellt.
In Smartphones ist ein kleines Magnetometer integriert, mit dem sich das Magnetfeld der Erde messen l�sst. Zudem bestimmt das Smartphone seine Position und Neigung anhand von anderer Sensorik. Das Ger�t kombiniert all diese Informationen und bestimmt somit die Richtung.

\subsection*{Geschwindigkeitsmessung}
Die Bewegungsgeschwindigkeit wird durch einen Be\-schleu\-ni\-gungs\-sen\-sor (auch Ac\-ce\-le\-ro\-me\-ter genannt) gemessen. Durch die auf ein Ger�t wirkende Tr�g\-heits\-kraft kann bestimmt werden ob  eine Geschwindigkeitszunahme oder -abnahme stattfindet. Anhand der mittleren Erdbeschleunigung l�sst sich diese dann genau berechnen.
In modernen Smartphones ist solch ein Sensor verbaut. Dieser l�sst sich in mobilen Applikationen dazu benutzen um die Geschwindigkeit des Nutzers zu ermitteln.

\subsection*{Thermometer}
Smartphones besitzen einen W�rmesensor, der genutzt werden kann um die Temperatur zu messen. Allerdings wird dieser meist durch die Betriebstemperatur beeinflusst und liefert daher nicht immer korrekte Werte f�r die Au�entemperatur. Aus diesem Grund werden die Messwerte von Thermometer Apps meist aus dem �rtlichen Wetterinformationsnetz bezogen. 
Die Innentemperatur hingegen wird tats�chlich durch den integrierten W�rmesensor bestimmt. Dazu muss das Ger�t sich aber eine gewisse Zeit im Standby Betrieb befinden, damit die Temperatur des Akkus das Ergebnis nicht verf�lscht.

\subsection*{Haptisches Feedback}
Haptisch bedeutet �f�hlbar� oder �ber�hrbar�. Bei Smartphones betrachten wir in diesem Zusammenhang den Vibrationsalarm. Dieser wird nativ dazu genutzt um dem Nutzer mitzuteilen, dass ein Anruf oder eine Nachricht eingegangen ist oder um Eingaben zu best�tigen.
Erzeugt werden die Vibrationen durch einen kleinen Motor, der durch eine Unwucht daf�r sorgt, dass das Geh�use vibriert. Bei einigen Ger�ten wird sogar der Lautsprecher dazu genutzt, indem er niedrigfrequente T�ne erzeugt, die das Geh�use in Schwingung versetzen.
