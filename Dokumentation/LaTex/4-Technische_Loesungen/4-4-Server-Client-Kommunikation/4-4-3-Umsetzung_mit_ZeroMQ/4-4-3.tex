%\subsection{Umsetzung mit ZeroMQ}


%F�r Echtzeit-Spiele ben�tigt man eine  permanente Daten�bertragung. Nicht nur weil langsame Updates irref�hrend sein k�nnen, sondern insbesondere wenn man mit mehreren Spielern gegeneinander spielt und die Versionen nicht synchron sind. Bei Verstecken etwa k�nnte ein um sechs Sekunden versp�tetes Update dazu f�hren, dass man nach einer Richtungs�nderung glaubt, in die richtige Richtung zu gehen, obwohl das Update vorher h�tte eintreffen sollen. Bei Snake kann man z.B. durch eine Schlange laufen, ohne eine Kollision zu bekommen, wenn die eigene Version �berzeugt ist, dass da noch keine Schlange ist, weil der Zustand des Spiels auf dem eigenen Smartphone noch nicht aktuell ist. 

%Generell kann man davon ausgehen, dass je schneller das Spiel ist und je mehr virtuelle Objekte einbezogen werden, desto h�ufiger und schneller m�ssen Updates kommen. Daher wollten wir eine �bertragung, die performant ist, schnell viele Nachrichten senden kann und au�erdem asynchron Nachrichten verarbeitet, also nicht jedes mal auf eine Antwort des Empf�ngers wartet, bevor eine weitere Nachricht abgeschickt werden kann.

%ZeroMQ kann all das. In unseren Probel�ufen war ZMQ (ZeroMQ) in der Lage, mehrere Tausend Nachrichten innerhalb einer Sekunde abzuschicken. ZMQ arbeitet standardm��ig asynchron. Au�erdem ist es recht einsteigerfreundlich. Das alles machte es f�r unsere Test-Apps zum idealen Kandidaten.


%Hier kommen Vergleiche zwischen ZMQ und seinen Konkurrenten (andere messaging Protokolle) hin.
