
%\subsection*{ZeroMQ}
%\subsection{Publish-Subscribe-Pattern}


%F�r unsere Spiele brauchten wir eine hohe Update-Geschwindigkeit, um das Spiel m�glichst synchron zu halten. Bei Snake kann es z.B. sonst passieren, dass man durch eine Schlange l�uft, ohne eine Kollision zu bekommen, weil der Zustand des Spiels auf dem eigenen Smartphone noch nicht aktuell ist und die andere Schlange noch nicht an der Position ist, an der sie momentan sein sollte. Bei unserem Flaggenspiel k�nnte es sonst vorkommen, dass man den Fahnentr�ger nicht markieren kann, weil er sich nicht an seiner physischen Position befindet.
%
%Generell kann man davon ausgehen, dass je schneller das Spiel ist und je mehr virtuelle Objekte einbezogen werden, desto h�ufiger und schneller m�ssen Updates kommen. Daher wollten wir eine �bertragung, die performant ist, schnell viele Nachrichten senden kann und au�erdem asynchron Nachrichten verarbeitet, also nicht jedes mal auf eine Antwort des Empf�ngers wartet, bevor eine weitere Nachricht abgeschickt werden kann.
%
%ZeroMQ kann all das. In unseren Probel�ufen war ZMQ (ZeroMQ) in der Lage, mehrere Tausend Nachrichten innerhalb einer Sekunde abzuschicken. ZMQ arbeitet standardm��ig asynchron. 



%Hier kommen Vergleiche zwischen ZMQ und seinen Konkurrenten (andere messaging Protokolle) hin.
