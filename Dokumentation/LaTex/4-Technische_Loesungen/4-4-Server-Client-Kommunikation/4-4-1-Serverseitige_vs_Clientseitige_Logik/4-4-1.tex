\subsection{Serverseitige vs Clientseitige Logik}
XMPP \footnote{http://xmpp.org/extensions/xep-0060.html}

Um den Zustand der einzelnen mobilen Ger�te zu synchronisieren und Chat-Nachrichten zu �bertragen, ist ein Server n�tig.
Hierzu bieten sich zwei unterschiedliche Modelle an.
Bei der klassischen Server-Client-Architektur �bernimmt der Server einen Gro�teil der Berechnungen.
Der Server h�lt einen zentralen, im Zweifelsfall g�ltigen Status
Die (Thin-)Clients �bertragen die Nutzereingaben und Sensordaten an den Server, der daraus einen neuen Zustand ermittelt und diesen den Clients mitteilt.
Alternativ bietet sich das Publish-Subscribe-Pattern an. 
Dabei h�lt der Server keinen Zustand, sondern leitet blo� Nachrichten von einem Client (Publisher) an einen anderen Client (Subscriber) weiter.
Im konkreten Fall eines interaktiven Spiels wird beispielsweise die Position eines Spielers an alle Spieler in der selben Spielinstanz weitergeleitet.
Die (Fat-)Clients m�ssen dabei alle Berechnungen �bernehmen. Dies fordert leistungsf�higere Endger�te. Diese haben sich als gen�gend leistungsf�hig erwiesen.
Bei fast jeder �nderung m�ssen alle Clients aktualisiert werden, da die Programmlogik redundant auf jedem Client vorliegt. Dies verringert Skalierbarkeit, Wartbarkeit und Erweiterbarkeit. In Echtzeitsystemen kommt zus�tzlich hinzu, dass, da es keinen definitiven, zentralen Zustand gibt, auf verschiedenen Clients zum gleichen Zeitpunkt unterschiedliche Zust�nde vorliegen. Da diese Zust�nde f�r Berechnungen genutzt werden, kann dies zu Unklarheiten und Fehlern f�hren, die abgefangen werden m�ssen.
Das Publish-Subscribe-Pattern bietet jedoch den Vorteil, dass es in diesem konkreten Fall Internet-Bandbreite spart, da kleinere Daten �bertragen werden m�ssen. Beispielsweise muss blo� der aktuelle Standpunkt eines Spielers �bertragen werden, nicht die daraus resultierenden Daten, die der Fat-Client selber berechnet.
Dieser Punkt war ausschlaggebend, da bei mobilen Endger�ten die Internet-Bandbreite eine stark limitierte Ressource ist.
Zus�tzlich verringert sich beim Publish-Subscribe-Pattern die Komplexit�t der Anwendung, da der Zustand nicht doppelt modelliert werden muss. Dies macht die Anwendung weniger fehleranf�llig 

\begin{center}
\begin{tabular}{l|l}
	Thin-Client & Fat-Client \\
	\hline
	 - Daten�bertragung & + Daten�bertragung \\
	 - Zustand doppelt & + Zustand nur auf Client \\
	 + zentraler, definitiver Zustand &  - verteilter Zustand \\
	 + Rechenleistung & - Rechenleistung \\
	 + Skalierbarkeit & - Skalierbarkeit \\
	 + Wartbarkeit & - Wartbarkeit \\
	 + Erweiterbarkeit & - Erweiterbarkeit \\
\end{tabular}
\end{center}
