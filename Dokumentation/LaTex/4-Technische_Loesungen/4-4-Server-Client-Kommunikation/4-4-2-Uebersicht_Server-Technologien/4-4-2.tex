\subsection{Server-Technologien}

%Die Kommunikation zwischen Client und Server kann als Push- oder Pull-Kommunikation realisiert werden. 
%Bei der Pull-Kommunikation fordert der Client die ben�tigten Informationen vom Server an. Im Falle einer Echtzeitanwendung muss dies in regelm��igen Abst�nden geschehen.
%Bei der Push-Kommunikation sendet der Server, wenn sich der Zustand �ndert, unaufgefordert Informationen an die betroffenen Clients.

%Pull-Kommunikation �ber HTTP-Requests umgesetzt hat eine nicht hinnehmbare Verz�gerung bewirkt.
%Die finale Implementation verwendet daher Push-Kommunikation. 
Eine spezialisierte Server-Architektur, die \textit{message passing} mit Push-Kom\-mu\-ni\-ka\-tion umsetzt, ist XMPP \cite{xmpp}. Es wurde jedoch eine allgemeine Server-L�sung bevorzugt, um die Flexibilit�t zu erh�hen und notfalls Berechnungen auf den Server auslagern zu k�nnen.
Im Falle mobiler Endger�te muss Push-Kommunikation �ber Sockets laufen, da diese nicht zuverl�ssig �ber eine IP-Adresse angesprochen werden k�nnen.
 Wir haben die Publish-Subscribe-Architektur jeweils mit Hilfe von WebSockets und ZeroMQ\footnote{\url{http://zeromq.org/}} umgesetzt und beide Ans�tze verglichen. 
Die auf WebSockets aufbauende L�sung haben wir mit Node.js\footnote{\url{https://nodejs.org/}} implementiert.
Die andere L�sung verwendet
JeroMQ\footnote{\url{https://github.com/zeromq/jeromq}}, eine Implementation von ZeroMQ in Java. 
  Wir haben uns f�r JeroMQ entschieden, da es so m�glich war, den Server in Java umzusetzen und so Server und Client in der selben Programmiersprache zu implementieren.
