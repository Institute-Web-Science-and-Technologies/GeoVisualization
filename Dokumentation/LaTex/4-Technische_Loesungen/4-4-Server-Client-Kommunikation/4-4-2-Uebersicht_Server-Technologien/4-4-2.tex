\subsection{�bersicht Server-Technologien}

Die Kommunikation zwischen Client und Server kann als Push- oder Pull-Kommunikation realisiert werden. 
Bei der Pull-Kommunikation fordert der Client die ben�tigten Informationen vom Server an. Im Falle einer Echtzeitanwendung muss dies in regelm��igen Abst�nden geschehen.
Bei der Push-Kommunikation sendet der Server, wenn sich der Zustand �ndert, unaufgefordert Informationen an die betroffenen Clients.
Im Falle mobiler Endger�te muss Push-Kommunikation �ber Sockets laufen, da diese Ger�te keine IP-Adresse haben.
Pull-Kommunikation �ber HTTP-Requests umgesetzt hat eine nicht hinnehmbare Verz�gerung bewirkt.
Die finale Implementation verwendet daher Push-Kommunikation. Eine spezialisierte Server-Architektur, die das Publish-Subscribe-Pattern und Push-Kommunikation verwendet ist XMPP \cite{XMPP}. Es wurde eine allgemeine Server-L�sung bevorzugt, um die Flexibilit�t zu erh�hen. Wir haben Node.js\footnote{http://nodejs.org/} und JeroMQ\footnote{https://github.com/zeromq/jeromq} verglichen. Wir haben uns f�r JeroMQ entschieden, um Server und Client in der selben Programmiersprache umsetzen zu k�nnen.
