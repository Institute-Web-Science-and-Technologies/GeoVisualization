\subsection{Kollisionsabfrage �ber Linienintersektion }
%Ob bei der Umsetzung von Snake die aktive Schlange mit einer anderen Schlange kollidiert ist, kann alternativ auch �ber Linienintersektion getestet werden. Hierbei wird zus�tzlich zum Kopf der aktiven Schlange auch der zweite Eckpunkt betrachtet. Genauer wird die Linie zwischen diesen beiden Punkten betrachtet. F�r jeden Eckpunkt einer passiven Schlange wird die Linie zwischen diesem und je dem n�chsten Eckpunkt betrachtet. Schneidet sich eine dieser Linien mit der vordersten Linie der aktiven Schlange, so sind beide Schlangen kollidiert. 

Bei der Kollisionsabfrage �ber Linienintersektion wird das erste Segment einer Schlange genommen und geschaut ob es sich mit einem beliebigem Segment einer anderen Schlange �berschneidet. Die Segmente werden bei uns jeweils �ber Punktepaare beschrieben. Aus den Punkten $P_{1}=\left( \begin{array}{c} x_{1} \\ y_{1} \end{array} \right)$ und $P_{2}=\left( \begin{array}{c} x_{2} \\ y_{2} \end{array} \right)$ berechnen wir die Koordinatenform $ax + by = c$ f�r die zur unserer Strecke geh�rende Gerade durch $a = y_{1}$ - $y_{2}$, $b = x_{2}$ - $x_{1}$, $c=x_{2}$$y_{1}$ - $x_{1}$$y_{2}$. Durch gleichsetzen der Koordinatenformen $a_{1}x + b_{1}y - c_{2} = 0$ und $a_{2}x + b_{2}y - c_{2} = 0$ der Segmente bekommen wir ein lineares Gleichungssystem welches wir f�r den Fall von zwei nicht parallelen Geraden direkt �ber die Cramersche Regel\cite{lenze2006basiswissen} l�sen: $S = \left( \begin{array}{c} x_{s} \\ y_{s} \end{array} \right)$, $x_s = \frac{c_{1}b_{2} - c_{2}b_{1}}{a_{1}b_{2} - a_{2}b_{1}}$, $y_s = \frac{a_{1}c_{2} - a_{2}c_{1}}{a_{1}b_{2} - a_{2}b_{1}}$. Im Fall von parallelen Geraden ist der Nenner 0 und wir werfen eine Exception. 
 Jetzt testen wir ob der Schnittpunkt $S$ der Geraden zwischen den Eckpunkten liegt, indem wir �berpr�fen ob der Punkt $S$ f�r beide Strecken jeweils zwischen den Eckpunkten der Strecke liegt.
%Jetzt testen wir ob der Schnittpunkt $S$ zwischen den Geraden auch Teil der Strecke ist indem wir �berpr�fen ob der x,y Wert von s jeweils zwischen den x,y Werten der beiden Punkte der Strecke liegt.
 



Diese Methode zur Kollisionsabfrage hat den Vorteil, dass sie eindeutige Kollisionen unabh�ngig von der Updategeschwindigkeit des GPS-Posi\-tions\-er\-mitt\-lungs\-diens\-tes erkennt.

\subsubsection{Entscheidung}
In diesem Projekt wurde sich f�r die Kollisionsabfrage �ber Abstandmessung entschieden. Mit dem Fokus darauf verschiedene Spiele umzusetzen wurde sich auf eine einheitliche Kollisionsabfrage geeinigt. In Tests mit der Abstandsmessung im Snakespiel waren zufriedenstellende Ergebnisse erzielt. Aus Gr�nden der Zeit- und Testersparnis wurde eine schon implementierte Funktion von Android verwendet \footnote{\url{http://developer.android.com/reference/android/location/Location.html}}. Hierbei wird von der Abstandsmessung die Erde als Ellipsoid ber�cksichtigt. Die Kollisionsabfrage �ber Linienintersektion w�re nur im Fall vom Snakespiel eine gute L�sung gewesen. 

