\subsection{Kollisionsabfrage �ber Linienintersektion }
%Bei der Kollisionsabfrage �ber Linienintersektion wird das erste Segment einer Schlange genommen und geschaut ob es sich mit einem beliebigem Segment einer anderen Schlange �berschneidet. Die Segmente werden bei uns jeweils �ber Punktepaare beschrieben. Aus den Punkten $P_{1}=\left( \begin{array}{c} x_{1} \\ y_{1} \end{array} \right)$ und $P_{2}=\left( \begin{array}{c} x_{2} \\ y_{2} \end{array} \right)$ berechnen wir die Koordinatenform ax + by = c f�r die zur unserer Strecke geh�rende Gerade durch a = $y_{1}$ - $y_{2}$, b = $x_{2}$ - $x_{1}$, c=$x_{2}$$y_{1}$ - $x_{1}$$y_{2}$. Durch gleichsetzen der Koordinatenformen $a_{1}$x + $b_{1}$y - $c_{2}$ = 0 und $a_{2}$x + $b_{2}$y - $c_{2}$ = 0 der Segmente bekommen wir ein lineares Gleichungssystem welches wir f�r den Fall von 2 nichtparallelen Geraden direkt �ber die Cramersche Regel l�sen: S = $\left( \begin{array}{c} x_{s} \\ y_{y} \end{array} \right)$, xs = $\frac{c_{1}b_{2} - c_{2}b_{1}}{a_{1}b_{2} - a_{2}b_{1}}$ ys = $\frac{a_{1}c_{2} - a_{2}c_{1}}{a_{1}b_{2} - a_{2}b_{1}}$. Im Fall von parallelen Geraden ist der Nenner 0 und wir werfen eine Exception. Jetzt testen wir ob der Schnittpunkt s zwischen den Geraden auch Teil der Strecke ist indem wir �berpr�fen ob der x,y Wert von s jeweils zwischen den x,y Werten der beiden Punkte der Strecke liegt.
 
Ob bei der Umsetzung von Snake die aktive Schlange mit einer anderen Schlange kollidiert ist, kann alternativ auch �ber Linienintersektion getestet werden. Hierbei wird zus�tzlich zum Kopf der aktiven Schlange auch das zweite Segment betrachtet. Genauer wird die Linie zwischen diesen beiden Segmenten betrachtet. F�r jedes Segment der passiven Schlange wird die Linie zwischen diesem und je dem n�chsten Segment betrachtet. Schneidet sich eine dieser Linien mit der vordersten Linie der aktiven Schlange, so sind beide Schlangen kollidiert. 

!! verwendete Methode? die oben genannte kann ich momentan nicht nachvollziehen und wir haben eine andere verwendet !!

Diese Methode zur Kollisionsabfrage hat den Vorteil, dass sie eindeutige Kollisionen unabh�ngig von der Updategeschwindigkeit des GPS-Posi\-tions\-er\-mitt\-lungs\-diens\-tes erkennt.