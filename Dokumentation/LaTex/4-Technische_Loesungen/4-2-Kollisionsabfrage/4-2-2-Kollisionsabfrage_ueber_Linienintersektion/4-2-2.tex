\subsection{Kollisionsabfrage �ber Linienintersektion }
Bei der Kollisionsabfrage �ber Linienintersektion wird das erste Segment einer Schlange genommen und geschaut ob es sich mit einem beliebigem Segment einer anderen Schlange �berschneidet. Die Segmente werden bei uns jeweils �ber Punktepaare beschrieben. Aus den Punkten P_{1}=$\left( \begin{array}{c} x_{1} \\ y_{1} \end{array} \right)$ und P_{2}=$\left( \begin{array}{c} x_{2} \\ y_{2} \end{array} \right)$ berechnen wir die Koordinatenform ax + by = c f�r die zur unserer Strecke geh�rende Gerade durch a = $y_{1}$ - $y_{2}$, b = $x_{2}$ - $x_{1}$, c=$x_{2}$$y_{1}$ - $x_{1}$$y_{2}$. Durch gleichsetzen der Koordinatenformen $a_{1}$x + $b_{1}$y - $c_{2}$ = 0 und $a_{2}$x + $b_{2}$y - $c_{2}$ = 0 der Segmente bekommen wir ein lineares Gleichungssystem welches wir f�r den Fall von 2 nichtparallelen Geraden direkt �ber die Cramersche Regel l�sen: S = $\left( \begin{array}{c} x_{s} \\ y_{y} \end{array} \right)$, xs = $\frac{c_{1}b_{2} - c_{2}b_{1}}{a_{1}b_{2} - a_{2}b_{1}}$ ys = $\frac{a_{1}c_{2} - a_{2}c_{1}}{a_{1}b_{2} - a_{2}b_{1}}$. Im Fall von parallelen Geraden ist der Nenner 0 und wir werfen eine Exception. Jetzt testen wir ob der Schnittpunkt s zwischen den Geraden auch Teil der Strecke ist indem wir �berpr�fen ob der x,y Wert von s jeweils zwischen den x,y Werten der beiden Punkte der Strecke liegt.
 
