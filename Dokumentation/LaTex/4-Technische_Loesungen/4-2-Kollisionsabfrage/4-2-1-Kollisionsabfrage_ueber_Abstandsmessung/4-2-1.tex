\subsection{Kollisionsabfrage �ber Abstandsmessung}
Bei der Kollisionsabfrage wird zwischen aktiven und passiven Objekten unterschieden. Objekte sind entweder beweglich oder unbeweglich und entweder solide oder unsolide. Jedes bewegliche Objekt nimmt, nachdem seine Positionsdaten ge�ndert wurden, die Rolle des aktiven Objekts ein. Das aktive Objekt wird dann mit jedem soliden Objekt verglichen und im Falle einer Kollision wird abh�ngig von den beteiligten Objekten ein entsprechender Effekt ausgel�st.


Bei der Kollisionsabfrage m�ssen beispielsweise f�r das Snake-Spiel vier F�lle unterscheiden werden: sowohl das aktive, wie auch das passive beteiligte Objekt k�nnen jeweils durch einen Kreis oder einen Polygonzug (eine Schlange) realisiert sein. Jeder Eckpunkt des Polygonzuges ist dabei ein Segment der Schlange.
Das aktive Objekt symbolisiert meist eine Kollision zwischen einem passiven Objekt und dem ausl�senden Spieler.
Das passive Objekt kann ein aufsammelbares Bonus-Objekt sein oder ein Objekt oder ein anderer Spieler.

Die Android API stellt eine Methode zur Verf�gung, die die Entfernung zwischen zwei Punkten, die �ber geographische Koordinaten bestimmt sind, berechnet \footnote{\url{http://developer.android.com/reference/android/location/Location.html}}. Damit kann die Kollisionsabfrage �ber Abstandsmessung implementiert werden.
Es wird eine Kollision ausgel�st, falls beide Objekte Kreise sind und die Entfernung zwischen ihren Kreismittelpunkten kleiner ist, als die Summe der Radien.

$|M_{aktiv}-M_{passiv}|<r_{aktiv}+r_{passiv}$

Bei einer Kollision mit einer Schlange als passives Objekt, wird dieses Kriterium auf alle Glieder der Schlange angewendet. Die Kollision zwischen dem aktiven Objekt und einem der Eckpunkte des Polygonzugs f�hrt zur Kollision mit der Schlange. 
Falls das aktive Objekt ein Polygonzug ist, m�ssen nur Kollisionen mit dessen erstem Element (dem Kopf der Schlange) ber�cksichtigt werden, da Kollisionen mit dem Schwanz der Schlange immer von anderen Objekten ausgel�st werden. Es handelt sich also um eine Kollision mit einem Kreis. Die weiteren F�lle sind analog zu den abgehandelten.

W�hrend Abstandsmessung als Kriterium f�r Kollision von Kreisen gut funktioniert, muss bei der Kollision von Polygonz�gen darauf geachtet werden, dass die Radien der Objekte gen�gend gro� gew�hlt werden, damit sich auch bei schwankender Genauigkeit der GPS-Werte und damit sehr unregelm��igen Abst�nden zwischen den Gliedern der Schlange, die Radien aufeinanderfolgender Glieder �berschneiden.

Im Experiment wurde bei guten Bedingungen Abweichungen der tat\-s�ch\-lich\-en Positionen von circa vier Metern gemessen (siehe Anhang \ref{app}).
In der Praxis hat sich aber ein Radius von zwei Metern als ausreichend erwiesen, da die Spieler die Diskrepanz zwischen realer und virtueller Welt bemerken und gegebenenfalls gegensteuern k�nnen. Auch ist ein Radius von zwei Metern fein genug, um das Hindurchlaufen durch eine Schlange zu verhindern.
Gleichzeitig bringt der geringere Radius den Vorteil, dass das Spiel auch auf kleineren Arealen gespielt werden kann und Spieler sch�rfere Kurven laufen k�nnen ohne Selbstkollisionen zu riskieren.