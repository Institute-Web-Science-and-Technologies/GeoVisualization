\subsection{Kollisionsabfrage �ber Abstandsmessung}
Die Android API stellt eine Methode zur Verf�gung, die die Entfernung zwischen zwei Punkten, die �ber geographische Koordinaten bestimmt sind, berechnet \footnote{\url{http://developer.android.com/reference/android/location/Location.html}}.
Bei der Kollisionsabfrage m�ssen wir f�r das Snake-Spiel vier F�lle unterscheiden: sowohl das aktive, wie auch das passive beteiligte Objekt k�nnen jeweils durch einen Kreis oder einen Polygonzug (eine Schlange) realisiert sein.
Das aktive Objekt symbolisiert meist den die Kollision mit dem passiven Objekt ausl�senden Spieler.
Das passive Objekt kann ein aufsammelbares Bonus-Objekt sein oder ein anderer Spieler.
Falls beide Objekte Kreise sind wird eine Kollision ausgel�st, falls die Entfernung zwischen beiden Kreismittelpunkten kleiner ist, als die Summe der Radien.

$|M_{aktiv}-M_{passiv}|<r_{aktiv}+r_{passiv}$

Bei einer Kollision mit einer Schlange, wird dieses Kriterium auf alle Glieder der Schlange angewendet. Die Kollision zwischen dem aktiven Objekt und einem der Eckpunkt des Polygonzugs f�hrt zur Kollision mit der Schlange. 
Falls das aktive Objekt ein Polygonzug ist, m�ssen nur Kollisionen mit dessen erstem Element (dem Kopf der Schlange) ber�cksichtigt werden, da Kollisionen mit Schwanz immer von anderen Objekten ausgel�st werden. Die weiteren F�lle sind analog zu den abgehandelten.
W�hrend Abstandsmessung als Kriterium f�r Kollision von Kreisen gut funktioniert, muss bei der Kollision von Polygonz�gen darauf geachtet werden, dass die Radien der Objekte gen�gend gro� gew�hlt werden, damit sich auch bei schwankender Genauigkeit der GPS-Werte und damit sehr unregelm��igen Abst�nden zwischen den Gliedern der Schlange, die Radien aufeinanderfolgender Glieder �berschneiden.
