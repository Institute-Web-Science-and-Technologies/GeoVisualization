\subsection{Kollisionsabfrage �ber Abstandsmessung}
Jedes Spiel verwendet unterschiedliche virtuelle und tats�chliche Objekte. Diese werden zur Kollisionsabfrage in mehrere Kategorien eingeteilt.
Objekte sind entweder beweglich oder unbeweglich und entweder solide oder unsolide.
Bei der Kollisionsabfrage k�nnen solide Objekte die passive Rolle und bewegliche Objekte die aktive Rolle einnehmen. 
Solide bewegliche Objekte k�nnen je nach Situation entweder aktiv oder passiv sein und unter Umst�nden auch mit sich selber kollidieren.
Jedes bewegliche Objekt nimmt, nachdem seine Positionsdaten ge�ndert wurden, die Rolle des aktiven Objekts ein. Das aktive Objekt wird dann mit jedem soliden Objekt verglichen, das dann die Rolle des passiven Objekts annimmt. Im Falle einer Kollision wird abh�ngig von den beteiligten Objekten ein entsprechender Effekt ausgel�st.

Die meisten Objekte haben Hitboxen in Form eines Kreises.
Die Android API stellt eine Methode zur Verf�gung, die die Entfernung zwischen zwei Punkten, die �ber geographische Koordinaten bestimmt sind, berechnet \footnote{\url{http://developer.android.com/reference/android/location/Location.html}}. Damit kann die Kollisionsabfrage �ber Abstandsmessung implementiert werden.
Es wird eine Kollision ausgel�st, falls beide Objekte Kreise sind und die Entfernung zwischen ihren Kreismittelpunkten kleiner ist, als die Summe der Radien.

$|M_{aktiv}-M_{passiv}|<r_{aktiv}+r_{passiv}$


Beim Snake-Spiel m�ssen hingegen vier F�lle unterscheiden werden: sowohl das aktive, wie auch das passive beteiligte Objekt k�nnen jeweils einen Kreis oder einen Kette von Kreisen (eine Schlange) als Hitbox haben. Die Schlange wird als Polygonzug von Punkten realisiert. Jeder Eckpunkt ist dabei der Mittelpunkt einer kreisf�rmigen Hitbox (Glied). Die Strecke zwischen zwei aufeinanderfolgenden Eckpunkten des Polygonzuges ist dabei ein Segment der Schlange.
Das aktive Objekt ist bei Snake immer ein Spieler, also eine Schlange.
Das passive Objekt kann ein aufsammelbares Bonus-Objekt, ein anderer Spieler oder der aktive Spieler selber sein.
Das aktive Objekt ist ein Polygonzug, trotzdem m�ssen nur Kollisionen mit dessen erstem Element (dem Kopf der Schlange) ber�cksichtigt werden, da Kollisionen mit dem Schwanz der Schlange immer von anderen Objekten ausgel�st werden
Bei einer Kollision mit einer Schlange als passives Objekt, wird jeder Eckpunkt des Polygonzuges auf Kollision mit dem aktiven Objekt gepr�ft. Die Kollision zwischen dem aktiven Objekt und einem Glied der passiven Schlange, also einem Kreis um einem Eckpunkte des Polygonzugs, f�hrt zur Kollision mit der Schlange. 

W�hrend Abstandsmessung als Kriterium f�r Kollision von Kreisen gut funktioniert, muss bei der Kollision von Polygonz�gen darauf geachtet werden, dass die Radien der kreisf�rmigen Hitboxen gen�gend gro� gew�hlt werden, damit sich auch bei schwankender Genauigkeit der GPS-Werte und damit sehr unregelm��igen Abst�nden zwischen den Gliedern der Schlange, die Hitboxen aufeinanderfolgender Glieder �berschneiden. 

Im Experiment wurde bei guten Bedingungen eine Abweichung der ermittelten Positionen von den tats�chlichen von circa vier Metern gemessen (siehe Anhang \ref{app}).
In der Praxis hat sich aber ein Radius von zwei Metern als ausreichend erwiesen, da die Spieler die Diskrepanz zwischen realer und virtueller Welt bemerken und gegebenenfalls gegensteuern k�nnen. Auch ist ein Radius von zwei Metern fein genug, um das Hindurchlaufen durch eine Schlange zu verhindern.
Gleichzeitig bringt der geringere Radius den Vorteil, dass das Spiel auch auf kleineren Arealen gespielt werden kann und Spieler sch�rfere Kurven laufen k�nnen ohne Selbstkollisionen zu riskieren.