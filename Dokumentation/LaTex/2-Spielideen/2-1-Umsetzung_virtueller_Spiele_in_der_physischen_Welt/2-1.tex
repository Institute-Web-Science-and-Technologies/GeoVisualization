\subsection{Umsetzung virtueller Spiele in der physischen Welt}

{\color{red}Definition virtuelle Spiele}

Eine Reihe von Spielideen, die in der virtuellen Welt sehr popul�r {\color{red}und geeignet f�r virtuelle Realit�t} sind.
\subsubsection{Snake}
Ein absoluter Klassiker, der fr�her auf keinem Handy fehlen durfte. Gerade wegen der
Popularit�t und Einfachheit recht interessant es in die physische Welt zu integrieren.
In einem begrenzten Areal gilt es eine Schlange geschickt zu steuern. Es m�ssen kleine
Items eingesammelt werden, wodurch die Schlange w�chst. Au�erdem muss darauf
geachtet werden, dass weder der Rand des Areals noch die eigene Schlange ber�hrt wird.
\subsubsection{Capture the Flag}
Nicht nur in der virtuellen Welt Spielbar, aber wohl dort erst richtig bekannt geworden.
Auch hier ist ein recht einfacher Spielmechanismus ausschlaggebend f�r einen leichten
Einsatz in der physischen Welt.
Zwei Teams treten gegeneinander an. Jedes Team besitzt eine Flagge, dessen Standort f�r
das gegnerische Team bekannt ist. Es wird versucht die Flagge des jeweils anderen Teams
zum Standort der eigenen Flagge zu bringen.
\subsubsection{Domination}
Meist in Kriegsszenarien integrierter virtueller Spielmodi, der sich mit wenig Aufwand in die
physische Welt �bertragen l�sst.
Zwei Teams treten gegeneinander an. Jedes Team hat einen Punktestand, der zu Anfang
gleich ist. Sinkt der Punktestand auf Null, so ist das Spiel f�r dieses Team verloren. Es gibt
verschiedene Areale, die eingenommen werden k�nnen. Hat ein Team mehr Areale
eingenommen als das andere, l�uft der Punktestand des Teams, das weniger Areale in
besitzt hat gegen Null.
