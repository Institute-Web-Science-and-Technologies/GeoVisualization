\section{Integration einer virtuellen Welt in Bewegungsspiele}



Nat�rlich k�nnen wir die Integration auch von der anderen Seite betrachten. Es kann in etablierte Bewegungsspiele der physischen Welt, eine virtuelle Umgebung eingef�gt werden.
Unter Bewegungsspielen verstehen wir jene, die sich ausschlie�lich in der physischen Welt abspielen und au�erdem mit �rtlicher Bewegung der Spieler zu tun haben.
Der GPS-Sensor, welchen wir haupts�chlich verwenden, liefert in H�usern nur sehr schlechte und �u�erst ungenaue Ergebnisse. Deswegen haben wir uns auf Bewegungsspiele beschr�nkt, die klassisch drau�en gespielt werden oder leicht an ein Freiluft-Spielfeld angepasst werden k�nnen, wie z.B. folgende Beispiele:

\subsubsection{Verstecken}
Es gibt eine Person ist der Sucher. Alle weiteren Mitspieler verstecken sich m�glichst gut und versuchen vom Suchenden nicht entdeckt zu werden. Hat der suchende Spieler alle Mitspieler gefunden ist das Spiel vorbei.

Um hier einen Mehrwert zu erlangen k�nnte f�r den Suchenden eine Richtungsangabe integriert werden, die anzeigt wo sich in etwa die anderen Mitspieler verstecken. Eine Anzeige, wer noch zu suchen ist und wer noch als versteckt gilt, kann ebenfalls zu mehr Spielvergn�gen beitragen. F�r die Spieler, die sich zu verstecken haben, ist interessant, wer der Suchende ist, was zus�tzlich angezeigt werden kann. Auch hier ist eine Punkteanzeige w�nschenswert.

\subsubsection{Fangen}
Ein Spieler versucht alle anderen Mitspieler durch Ber�hren zu �fangen�. Hat er geschafft einen Mitspieler zu ber�hren ist nun dieser Mitspieler seinerseits dran einen anderen zu fangen. Der bis dato fangende Spieler wird gleichzeitig zum Gejagten.

M�gliche Erweiterungen zum klassischen Spiel w�ren da zum Beispiel eine Anzeige, welcher Mitspieler der F�nger ist, wer wie oft bisher gefangen wurde, eine Karte, die anzeigt wo sie die Mitspieler befinden oder gar eine Kollisionsabfrage, mit der man die Gegenspieler zum Fangen nicht mehr ber�hren muss sondern lediglich in einen vorher festgelegten Umkreis gelangen muss.

\subsubsection{Topfschlagen}
Ein Spieler bekommt die Augen verbunden. Er versucht anhand von Tipps der Mitspieler (\glqq n�her dran\grqq oder \glqq weiter weg\grqq) einen Gegenstand zu finden. Hat er dies geschafft, beendet er somit seine Runde und der n�chste Spieler ist dran. Er bekommt seinerseits die Augen verbunden und ein neuer Gegenstand wird platziert.

Eine Aufwertung des Spiels kann hier eine Zeitmessung sein, wie lange die einzelnen Spieler f�r das Finden des Gegenstands gebraucht haben. Au�erdem k�nnen auditive (zum Beispiel ein sich steigerndes Piepsger�usch) oder haptische (st�rkere/schw�chere Vibration) Hilfsmittel einen Mehrwert bilden.