\section{Integration einer virtuellen Welt in Bewegungsspiele}

Unter Bewegungsspielen verstehen wir jene, die sich ausschlie�lich in der physischen Welt abspielen und au�erdem mit �rtlicher Bewegung der Spieler zu tun haben.
\newline

Nat�rlich k�nnen wir die Integration auch von der anderen Seite betrachten. Es kann in
etablierte Bewegungsspiele der physischen Welt, eine virtuelle Umgebung eingef�gt
werden.
Es folgen simple Spielevorschl�ge:
\subsubsection{Verstecken}
Es gibt eine Person, die der Suchende ist. Alle weiteren Mitspieler verstecken sich m�glichst
gut und versuchen vom Suchenden nicht entdeckt zu werden. Hat der suchende Spieler
alle Mitspieler gefunden ist das Spiel vorbei.
\subsubsection{Fangen}
Ein Spieler versucht alle anderen Mitspieler durch ber�hren zu �fangen�. Hat er geschafft
einen Mitspieler zu ber�hren ist nun dieser Mitspieler seinerseits dran einen anderen zu
fangen.
\subsubsection{Topfschlagen}
Ein Spieler bekommt die Augen verbunden. Er versucht anhand von Tipps der Mitspieler
(�n�her dran� oder �weiter weg�) einen Gegenstand zu finden.

